% preamble.tex - Gemeinsame Präambel für alle Übungsblätter
\documentclass[12pt,a4paper]{article}
\usepackage[utf8]{inputenc}
\usepackage[ngerman]{babel}
\usepackage{amsmath,amssymb}
\usepackage{geometry}
\usepackage{enumitem}
%\usepackage{enumerate}
\usepackage{tikz}
\usepackage{array}
\usepackage{multirow}
\usepackage{graphicx}
\usepackage{fancyhdr}
\usepackage{lastpage}
\usepackage{pgfplots}
\usepackage{xcolor}
\usepackage{tcolorbox}
\pgfplotsset{compat=1.17}

% Seitenränder
\geometry{margin=2cm}

% Kopf- und Fußzeilen-Stil
\pagestyle{fancy}
\fancyhf{}
\renewcommand{\headrulewidth}{0.5pt}
\renewcommand{\footrulewidth}{0.5pt}

% Standard Kopf- und Fußzeile (kann in jedem Blatt überschrieben werden)
\lhead{Mathematik 6. Klasse - Mittelschule}
\chead{}
\rhead{Datum: \underline{\hspace{3cm}}}
\lfoot{Name: \underline{\hspace{5cm}}}

% Anpassungen für bessere Lesbarkeit
\setlength{\parindent}{0pt}
\setlength{\parskip}{0.5em}

% Eigene Befehle für häufig verwendete Elemente
\newcommand{\antwortlinie}[1]{\underline{\hspace{#1}}}
\newcommand{\rechnung}{\vspace{0.3cm}Rechnung: }
\newcommand{\antwort}{\vspace{0.3cm}Antwort: }

% Farben für Hervorhebungen
\definecolor{hinweis}{RGB}{0,102,204}
\definecolor{wichtig}{RGB}{204,0,0}

% Befehl für Hinweisboxen
\newcommand{\hinweisbox}[1]{%
\vspace{0.5cm}
\noindent\fbox{\parbox{0.95\textwidth}{\textit{#1}}}
\vspace{0.5cm}
}

% Befehl für Motivationsboxen am Ende
\newcommand{\motivationsbox}[1]{%
\vspace{1cm}
\begin{center}
\fbox{\Large #1}
\end{center}
}

\newtcolorbox{merkbox}[1][]{
    colback=blue!5!white,
    colframe=blue!50!black,
    fonttitle=\bfseries,
    title=#1
}

\newtcolorbox{beispielbox}[1][]{
    colback=green!5!white,
    colframe=green!50!black,
    fonttitle=\bfseries,
    title=#1
}


\newcommand{\blattnummer}{1}
\newcommand{\blattthema}{Rationale Zahlen}
\newcommand{\bearbeitungszeit}{90}


\begin{document}
% Überschrift mit Blattnummer und Thema
\section*{\centering Übungsblatt \blattnummer: \blattthema}
\subsection*{\centering Bearbeitungszeit: \bearbeitungszeit\ Minuten}

% Optionale Lernziele (kann definiert werden)
\ifdefined\lernziele
\hinweisbox{\textbf{Lernziele:} \lernziele}
\fi

% Optionale Hilfsmittel (kann definiert werden)
\ifdefined\hilfsmittel
\noindent\textbf{Erlaubte Hilfsmittel:} \hilfsmittel
\vspace{0.5cm}
\fi

% Optionale Bearbeitungshinweise
\ifdefined\bearbeitungshinweise
\noindent\textit{\bearbeitungshinweise}
\vspace{0.5cm}
\fi

\subsection*{Teil A: Brüche verstehen und darstellen (20 Minuten)}

\begin{enumerate}[label=\arabic*.]
    \item \textbf{Welcher Bruch ist dargestellt? Schreibe auf:}
    \vspace{0.5cm}

    \textit{Stelle dir vor: Ein Kreis ist in 4 gleiche Teile geteilt, 3 sind gefärbt}

    \begin{tabular}{ll}
        a) 3 von 4 Teilen = $\dfrac{\phantom{00}}{\phantom{00}}$ & b) 1 von 2 Teilen = $\dfrac{\phantom{00}}{\phantom{00}}$ \\[2ex]
        c) 5 von 8 Teilen = $\dfrac{\phantom{00}}{\phantom{00}}$ & d) 2 von 3 Teilen = $\dfrac{\phantom{00}}{\phantom{00}}$
    \end{tabular}

    \vspace{1cm}

    \item \textbf{Erweitere die Brüche:} 
    \vspace{0.5cm}
    
    \begin{tabular}{ll}
        a) $\dfrac{1}{2} = \dfrac{1 \cdot 2}{2 \cdot 2} = \dfrac{\phantom{00}}{4}$ & b) $\dfrac{1}{3} = \dfrac{1 \cdot 2}{3 \cdot 2} = \dfrac{\phantom{00}}{6}$ \\[4ex]
        c) $\dfrac{2}{5} = \dfrac{2 \cdot 2}{5 \cdot 2} = \dfrac{\phantom{00}}{10}$ & d) $\dfrac{3}{4} = \dfrac{3 \cdot 3}{4 \cdot 3} = \dfrac{\phantom{00}}{12}$
    \end{tabular}

    \vspace{1cm}

    \item \textbf{Kürze die Brüche so weit wie möglich:} 
    \vspace{0.5cm}

    \begin{tabular}{ll}
        a) $\dfrac{6}{8} = \dfrac{6:2}{8:2} = \dfrac{\phantom{00}}{\phantom{00}}$ & b) $\dfrac{9}{12} = \dfrac{9:3}{12:3} = \dfrac{\phantom{00}}{\phantom{00}}$ \\[4ex]
        c) $\dfrac{10}{15} = \dfrac{10:5}{15:5} = \dfrac{\phantom{00}}{\phantom{00}}$ & d) $\dfrac{14}{21} = \dfrac{14:7}{21:7} = \dfrac{\phantom{00}}{\phantom{00}}$
    \end{tabular}
\end{enumerate}

\newpage
\subsection*{Teil B: Addition und Subtraktion von Brüchen (30 Minuten)}

\begin{enumerate}[label=\arabic*.]
    \item \textbf{Addiere die gleichnamigen Brüche:}
    \vspace{0.5cm}
    \begin{tabular}{ll}
        a) $\dfrac{2}{7} + \dfrac{3}{7} = \dfrac{2+3}{7} = \dfrac{\phantom{00}}{7}$ = \underline{\hspace{3cm}} &
        b) $\dfrac{1}{5} + \dfrac{2}{5} = \dfrac{\phantom{00}}{\phantom{00}}$ = \underline{\hspace{3cm}} \\[4ex]
        c) $\dfrac{3}{8} + \dfrac{4}{8} = \dfrac{\phantom{00}}{\phantom{00}}$ = \underline{\hspace{3cm}} &
        d) $\dfrac{2}{9} + \dfrac{5}{9} = \dfrac{\phantom{00}}{\phantom{00}}$ = \underline{\hspace{3cm}}
    \end{tabular}

    \vspace{1cm}

    \item \textbf{Subtrahiere die gleichnamigen Brüche:}
    \vspace{0.5cm}
    \begin{tabular}{ll}
        a) $\dfrac{5}{6} - \dfrac{2}{6} = \dfrac{5-2}{6} = \dfrac{\phantom{00}}{6}$ = \underline{\hspace{3cm}} &
        b) $\dfrac{7}{8} - \dfrac{3}{8} = \dfrac{\phantom{00}}{\phantom{00}}$ = \underline{\hspace{3cm}} \\[4ex]
        c) $\dfrac{4}{5} - \dfrac{1}{5} = \dfrac{\phantom{00}}{\phantom{00}}$ = \underline{\hspace{3cm}} &
        d) $\dfrac{8}{10} - \dfrac{3}{10} = \dfrac{\phantom{00}}{\phantom{00}}$ = \underline{\hspace{3cm}}
    \end{tabular}

    \vspace{1cm}

    \item \textbf{Mache gleichnamig und rechne:}
    \textit{Beispiel:} $\dfrac{1}{2} + \dfrac{1}{3} = \dfrac{3}{6} + \dfrac{2}{6} = \dfrac{5}{6}$
    \vspace{0.5cm}
    \begin{enumerate}[label=\alph*)]
        \item $\dfrac{1}{2} + \dfrac{1}{4} = \dfrac{\phantom{00}}{4} + \dfrac{1}{4} = \dfrac{\phantom{00}}{4}$ = \underline{\hspace{3cm}}
        \vspace{0.5cm}
        \item $\dfrac{2}{3} + \dfrac{1}{6} = \dfrac{\phantom{00}}{6} + \dfrac{1}{6} = \dfrac{\phantom{00}}{6}$ = \underline{\hspace{3cm}}
        \vspace{0.5cm}
        \item $\dfrac{3}{4} - \dfrac{1}{2} = \dfrac{3}{4} - \dfrac{\phantom{00}}{4} = \dfrac{\phantom{00}}{4}$ = \underline{\hspace{3cm}}
    \end{enumerate}
\end{enumerate}

\newpage
\subsection*{Teil C: Dezimalzahlen und Umwandlungen (25 Minuten)}

\begin{enumerate}[label=\arabic*.]
    \item \textbf{Wandle Brüche in Dezimalzahlen um:}
    \textit{Tipp: Teile Zähler durch Nenner}
    \vspace{0.5cm}
    \begin{tabular}{ll}
        a) $\dfrac{1}{2} = 1 : 2 =$ \underline{\hspace{3cm}} & b) $\dfrac{1}{4} = 1 : 4 =$ \underline{\hspace{3cm}} \\[2ex]
        c) $\dfrac{3}{4} = 3 : 4 =$ \underline{\hspace{3cm}} & d) $\dfrac{1}{5} = 1 : 5 =$ \underline{\hspace{3cm}} \\[2ex]
        e) $\dfrac{3}{10} = 3 : 10 =$ \underline{\hspace{3cm}} & f) $\dfrac{7}{10} = 7 : 10 =$ \underline{\hspace{3cm}}
    \end{tabular}

    \vspace{1cm}

    \item \textbf{Wandle Dezimalzahlen in Brüche um:}
    \textit{Tipp: 0,5 = $\dfrac{5}{10} = \dfrac{1}{2}$}
    \vspace{0.5cm}
        \begin{tabular}{ll}
        a) $0{,}5 = \dfrac{5}{10} = \dfrac{\phantom{00}}{\phantom{00}}$ & b) $0{,}25 = \dfrac{25}{100} = \dfrac{\phantom{00}}{\phantom{00}}$ \\[3ex]
        c) $0{,}8 = \dfrac{8}{10} = \dfrac{\phantom{00}}{\phantom{00}}$ & d) $0{,}75 = \dfrac{75}{100} = \dfrac{\phantom{00}}{\phantom{00}}$ \\[3ex]
        e) $0{,}6 = \dfrac{6}{10} = \dfrac{\phantom{00}}{\phantom{00}}$ & f) $0{,}2 = \dfrac{2}{10} = \dfrac{\phantom{00}}{\phantom{00}}$
    \end{tabular}

    \vspace{1cm}

    \item \textbf{Vergleiche und setze das richtige Zeichen ein:} $<$, $>$ oder $=$
    \vspace{0.5cm}
    \begin{tabular}{ll}
        a) $\dfrac{1}{2}$ \underline{\hspace{1cm}} $0{,}5$ & b) $\dfrac{3}{4}$ \underline{\hspace{1cm}} $0{,}8$ \\[2ex]
        c) $0{,}25$ \underline{\hspace{1cm}} $\dfrac{1}{4}$ & d) $\dfrac{2}{5}$ \underline{\hspace{1cm}} $0{,}4$
    \end{tabular}
\end{enumerate}

\newpage
\subsection*{Teil D: Sachaufgaben (15 Minuten)}

\begin{enumerate}[label=\arabic*.]
    \item \textbf{Lisa und Tom teilen sich einen Kuchen:}
    \vspace{0.5cm}

    Lisa isst $\dfrac{1}{4}$ des Kuchens, Tom isst $\dfrac{2}{8}$ des Kuchens.

    \vspace{0.5cm}
    \begin{enumerate}[label=\alph*)]
        \item Wandle Toms Anteil in Vierteln um: $\dfrac{2}{8} = \dfrac{\phantom{00}}{\phantom{00}}$
        \vspace{0.5cm}
        \item Wer hat mehr gegessen? \underline{\hspace{6cm}}
        \vspace{0.5cm}
        \item Wie viel haben beide zusammen gegessen? 

        Rechnung: $\dfrac{1}{4} + \dfrac{2}{8} = \dfrac{1}{4} + \dfrac{\phantom{00}}{4} = \dfrac{\phantom{00}}{4}$ = \underline{\hspace{3cm}}
    \end{enumerate}

    \vspace{1cm}

    \item \textbf{Wasserglas-Aufgabe:}
    \vspace{0.5cm}

    Ein Wasserglas ist zu $\dfrac{3}{5}$ gefüllt.

    \vspace{0.5cm}
    \begin{enumerate}[label=\alph*)]
        \item Wandle in Dezimalzahl um: $\dfrac{3}{5} = 3 : 5 =$ \underline{\hspace{3cm}}
        \vspace{0.5cm}
        \item Wandle in Prozent um: \underline{\hspace{3cm}} $\%$
        \vspace{0.5cm}
        \item Wie viel fehlt noch bis das Glas voll ist?

        Rechnung: $1 - \dfrac{3}{5} = \dfrac{5}{5} - \dfrac{3}{5} = \dfrac{\phantom{00}}{5}$ = \underline{\hspace{3cm}}
    \end{enumerate}
\end{enumerate}

% footer_template.tex - Vorlage für das Ende jedes Blattes
% Verwendung:
% \newcommand{\motivationstext}{Geschafft! Kontrolliere nochmal deine Lösungen!}
% \input{footer_template}

% Motivationsbox am Ende (wenn definiert)
\motivationsbox{Geschafft! Kontrolliere nochmal deine Lösungen!}

% Optionale Zusatzinformationen
\ifdefined\zusatzinfo
\vspace{0.5cm}
\begin{center}
\zusatzinfo
\end{center}
\fi

% Optionaler Platz für Notizen
\ifdefined\notizbereich
\vspace{1cm}
\noindent\textbf{Notizen / Fragen:}
\vspace{0.3cm}

\noindent\rule{\textwidth}{0.5pt}
\vspace{0.3cm}

\noindent\rule{\textwidth}{0.5pt}
\vspace{0.3cm}

\noindent\rule{\textwidth}{0.5pt}
\vspace{0.3cm}

\noindent\rule{\textwidth}{0.5pt}
\fi

% Optionale Selbsteinschätzung
\ifdefined\selbsteinschaetzung
\vspace{1cm}
\noindent\textbf{Selbsteinschätzung:} Wie gut konntest du die Aufgaben lösen?

\vspace{0.3cm}
\begin{center}
\begin{tabular}{|l|c|c|c|c|}
\hline
& sehr gut & gut & geht so & schwierig \\
\hline
Teil A & $\square$ & $\square$ & $\square$ & $\square$ \\
\hline
Teil B & $\square$ & $\square$ & $\square$ & $\square$ \\
\hline
Teil C & $\square$ & $\square$ & $\square$ & $\square$ \\
\hline
Teil D & $\square$ & $\square$ & $\square$ & $\square$ \\
\hline
\end{tabular}
\end{center}
\fi


\end{document}