% preamble.tex - Gemeinsame Präambel für alle Übungsblätter
\documentclass[12pt,a4paper]{article}
\usepackage[utf8]{inputenc}
\usepackage[ngerman]{babel}
\usepackage{amsmath,amssymb}
\usepackage{geometry}
\usepackage{enumitem}
%\usepackage{enumerate}
\usepackage{tikz}
\usepackage{tikz-3dplot}
\usepackage{array}
\usepackage{tabularx}
\usepackage{multirow}
\usepackage{graphicx}
\usepackage{fancyhdr}
\usepackage{lastpage}
\usepackage{pgfplots}
\usepackage{xcolor}
\usepackage{tcolorbox}
\pgfplotsset{compat=1.17}

% Seitenränder
\geometry{margin=2cm}

% Kopf- und Fußzeilen-Stil
\pagestyle{fancy}
\fancyhf{}
\renewcommand{\headrulewidth}{0.5pt}
\renewcommand{\footrulewidth}{0.5pt}

% Standard Kopf- und Fußzeile (kann in jedem Blatt überschrieben werden)
\lhead{Mathematik Ferienunterricht}
\chead{}
\rhead{Datum: \underline{\hspace{3cm}}}
\lfoot{Name: \underline{\hspace{5cm}}}
\rfoot{Benjamin Chladni}

% Anpassungen für bessere Lesbarkeit
\setlength{\parindent}{0pt}
\setlength{\parskip}{0.5em}

% Eigene Befehle für häufig verwendete Elemente
\newcommand{\antwortlinie}[1]{\underline{\hspace{#1}}}
\newcommand{\rechnung}{\vspace{0.3cm}Rechnung: }
\newcommand{\antwort}{\vspace{0.3cm}Antwort: }

% Farben für Hervorhebungen
\definecolor{hinweis}{RGB}{0,102,204}
\definecolor{wichtig}{RGB}{204,0,0}

% Befehl für Hinweisboxen
\newcommand{\hinweisbox}[1]{%
\vspace{0.5cm}
\noindent\fbox{\parbox{0.95\textwidth}{\textit{#1}}}
\vspace{0.5cm}
}

% Befehl für Motivationsboxen am Ende
\newcommand{\motivationsbox}[1]{%
\vspace{1cm}
\begin{center}
\fbox{\Large #1}
\end{center}
}

\newtcolorbox{merkbox}[1][]{
    colback=blue!5!white,
    colframe=blue!50!black,
    fonttitle=\bfseries,
    title=#1
}

\newtcolorbox{beispielbox}[1][]{
    colback=green!5!white,
    colframe=green!50!black,
    fonttitle=\bfseries,
    title=#1
}

\newtcolorbox{warnbox}[1][]{
    colback=red!5!white,
    colframe=red!50!black,
    fonttitle=\bfseries,
    title=#1
}

\newtcolorbox{zielbox}[1][]{
    colback=purple!5!white,
    colframe=purple!50!black,
    fonttitle=\bfseries,
    title=#1
}