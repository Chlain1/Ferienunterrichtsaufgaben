\subsection*{Teil C: Proportionalitäten und Prozentrechnung (25 Minuten)}

\begin{enumerate}[label=\arabic*.]

    \item \textbf{Dreisatz (direkt proportional):}
    \textit{8 Brötchen kosten 4€. Was kosten 12 Brötchen?}
    \vspace{0.5cm}

    \begin{tabular}{ll}
        8 Brötchen $\rightarrow$ 4€ & \\
        1 Brötchen $\rightarrow$ \underline{\hspace{1cm}}€ & \\
        12 Brötchen $\rightarrow$ \underline{\hspace{1cm}}€ &
    \end{tabular}

    \vspace{1cm}

    \item \textbf{Dreisatz (indirekt proportional):}
    \textit{4 Maler streichen ein Haus in 12 Tagen. Wie lange brauchen 6 Maler?}
    \vspace{0.5cm}

    \begin{tabular}{ll}
        4 Maler $\rightarrow$ 12 Tage & Gesamtarbeit: $4 \cdot 12 = \phantom{00}$ \\
        1 Maler $\rightarrow$ \underline{\hspace{1cm}} Tage & \\
        6 Maler $\rightarrow$ \underline{\hspace{1cm}} Tage &
    \end{tabular}

    \vspace{1cm}

    \item \textbf{Prozentrechnung:}
    \vspace{0.5cm}

    \begin{enumerate}[label=\alph*)]
        \item Ein Fahrrad kostet 300€. Es gibt 15\% Rabatt. Neuer Preis: \underline{\hspace{3cm}}
        \vspace{0.3cm}

        \item 24€ sind 20\% von welchem Betrag? Grundwert: \underline{\hspace{3cm}}
    \end{enumerate}

\end{enumerate}