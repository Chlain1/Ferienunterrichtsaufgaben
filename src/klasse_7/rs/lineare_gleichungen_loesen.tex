\subsection*{Teil B: Lineare Gleichungen lösen (35 Minuten)}

\begin{enumerate}[label=\arabic*.]

    \item \textbf{Einfache Gleichungen der Form ax = b:}
    \textit{Beispiel:} $3x = 12$ $|:3$ $\Rightarrow$ $x = 4$
    \vspace{0.5cm}

    \begin{tabular}{ll}
        a) $2x = 8$ $|:\phantom{0}$ $\Rightarrow$ $x$ = \underline{\hspace{2cm}} & b) $5x = 25$ $|:\phantom{0}$ $\Rightarrow$ $x$ = \underline{\hspace{2cm}} \\[3ex]
        c) $4x = 20$ $|:\phantom{0}$ $\Rightarrow$ $x$ = \underline{\hspace{2cm}} & d) $7x = 21$ $|:\phantom{0}$ $\Rightarrow$ $x$ = \underline{\hspace{2cm}}
    \end{tabular}

    \vspace{1cm}

    \item \textbf{Gleichungen der Form x + b = c:}
    \textit{Beispiel:} $x + 5 = 12$ $|-5$ $\Rightarrow$ $x = 7$
    \vspace{0.5cm}

    \begin{tabular}{ll}
        a) $x + 3 = 10$ $|\phantom{00}$ $\Rightarrow$ $x$ = \underline{\hspace{2cm}} & b) $x - 4 = 7$ $|\phantom{00}$ $\Rightarrow$ $x$ = \underline{\hspace{2cm}} \\[3ex]
        c) $x + 6 = 15$ $|\phantom{00}$ $\Rightarrow$ $x$ = \underline{\hspace{2cm}} & d) $x - 2 = 9$ $|\phantom{00}$ $\Rightarrow$ $x$ = \underline{\hspace{2cm}}
    \end{tabular}

    \vspace{1cm}

    \item \textbf{Gleichungen der Form ax + b = c:}
    \textit{Beispiel:} $2x + 3 = 11$ $|-3$ $\Rightarrow$ $2x = 8$ $|:2$ $\Rightarrow$ $x = 4$
    \vspace{0.5cm}

    \begin{enumerate}[label=\alph*)]
        \item $3x + 2 = 14$ \\
        $3x + 2 = 14$ $|\phantom{00}$ \\
        $3x = \phantom{00}$ $|\phantom{00}$ \\
        $x = \phantom{00}$
        \vspace{0.5cm}

        \item $2x - 5 = 7$ \\
        $2x - 5 = 7$ $|\phantom{00}$ \\
        $2x = \phantom{00}$ $|\phantom{00}$ \\
        $x = \phantom{00}$
        \vspace{0.5cm}

        \item $4x + 1 = 17$ \\
        $4x + 1 = 17$ $|\phantom{00}$ \\
        $4x = \phantom{00}$ $|\phantom{00}$ \\
        $x = \phantom{00}$
        \vspace{0.5cm}

        \item $5x - 3 = 22$ \\
        $5x - 3 = 22$ $|\phantom{00}$ \\
        $5x = \phantom{00}$ $|\phantom{00}$ \\
        $x = \phantom{00}$
    \end{enumerate}

\end{enumerate}