\subsection*{Teil B: Gleichungen und Geometrie (25 Minuten)}

\begin{enumerate}[label=\arabic*.]

    \item \textbf{Gleichung lösen:}
    \vspace{0.5cm}

    $3x - 7 = 2x + 5$ \\
    $3x - 7 = 2x + 5$ $|\phantom{00}$ \\
    $3x - 2x = 5 + 7$ \\
    $x = \phantom{00}$

    \vspace{0.5cm}
    Probe: $3 \cdot \phantom{00} - 7 = \phantom{00}$ und $2 \cdot \phantom{00} + 5 = \phantom{00}$ \checkmark

    \vspace{1cm}

    \item \textbf{Parallelverschiebung:}
    \vspace{0.5cm}

    \begin{enumerate}[label=\alph*)]
        \item Punkt A(2|5) wird um den Vektor $\begin{pmatrix} 3 \\ -2 \end{pmatrix}$ verschoben:
        A'(\underline{\hspace{1cm}}|\underline{\hspace{1cm}})
        \vspace{0.5cm}

        \item Mittelpunkt von B(1|3) und C(7|1):
        M(\underline{\hspace{1cm}}|\underline{\hspace{1cm}})
    \end{enumerate}

    \vspace{1cm}

    \item \textbf{Winkel berechnen:}
    \vspace{0.5cm}

    \begin{enumerate}[label=\alph*)]
        \item In einem Dreieck: $\alpha = 70°$, $\beta = 45°$, $\gamma$ = \underline{\hspace{2cm}}
        \vspace{0.3cm}

        \item An parallelen Geraden: Stufenwinkel zu 125° = \underline{\hspace{2cm}}
    \end{enumerate}

\end{enumerate}