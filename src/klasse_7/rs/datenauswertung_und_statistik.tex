\subsection*{Teil C: Datenauswertung und Statistik (35 Minuten)}

\begin{enumerate}[label=\arabic*.]

    \item \textbf{Statistische Kenngrößen berechnen:}
    \textit{Gegeben: Körpergrößen von 7 Schülern (in cm): 160, 165, 158, 170, 162, 168, 165}
    \vspace{0.5cm}

    \begin{enumerate}[label=\alph*)]
        \item \textbf{Ordne die Werte:} \underline{\hspace{8cm}}
        \vspace{0.5cm}

        \item \textbf{Arithmetisches Mittel:} $\overline{x} = \dfrac{\text{Summe aller Werte}}{\text{Anzahl}}$
        Summe: $160 + 165 + 158 + 170 + 162 + 168 + 165 = \phantom{000}$
        $\overline{x} = \dfrac{\phantom{000}}{7} = \phantom{00}$ cm
        \vspace{0.5cm}

        \item \textbf{Zentralwert (Median):} Der mittlere Wert: \underline{\hspace{2cm}} cm
        \vspace{0.5cm}

        \item \textbf{Modalwert:} Der häufigste Wert: \underline{\hspace{2cm}} cm
        \vspace{0.5cm}

        \item \textbf{Spannweite:} Größter - kleinster Wert: $\phantom{000} - \phantom{000} = \phantom{00}$ cm
    \end{enumerate}

    \vspace{1cm}

    \item \textbf{Diagramme interpretieren:}
    \vspace{0.5cm}

    \begin{enumerate}[label=\alph*)]
        \item Eine Umfrage ergab: 40\% mögen Fußball, 30\% Basketball, 20\% Tennis, 10\% andere Sportarten.
        Bei 200 befragten Schülern:
        \begin{itemize}
            \item Fußball: $200 \cdot 0,4 = \phantom{00}$ Schüler
            \item Basketball: $200 \cdot 0,3 = \phantom{00}$ Schüler
            \item Tennis: $200 \cdot 0,2 = \phantom{00}$ Schüler
            \item Andere: $200 \cdot 0,1 = \phantom{00}$ Schüler
        \end{itemize}
        \vspace{0.5cm}

        \item In einem manipulativen Diagramm wird die y-Achse nicht bei 0 begonnen.
        Warum ist das problematisch? \underline{\hspace{8cm}}
        \underline{\hspace{12cm}}
    \end{enumerate}

    \vspace{1cm}

    \item \textbf{Stichproben beurteilen:}
    \vspace{0.5cm}

    \begin{enumerate}[label=\alph*)]
        \item Für eine Umfrage über Lieblingsfächer werden nur Schüler aus der Mathe-AG befragt.
        Ist diese Stichprobe repräsentativ? \underline{\hspace{2cm}}
        Begründung: \underline{\hspace{8cm}}
        \vspace{0.5cm}

        \item Eine Umfrage über Fernsehgewohnheiten wird nur bei 16-18 Jährigen durchgeführt.
        Für welche Grundgesamtheit ist sie repräsentativ? \underline{\hspace{6cm}}
    \end{enumerate}

\end{enumerate}