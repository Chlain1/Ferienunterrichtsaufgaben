\subsection*{Teil A: Grundlagen der Potenzen (25 Minuten)}

\begin{enumerate}[label=\arabic*.]

    \item \textbf{Schreibe als Potenz:}
    \vspace{0.5cm}

    \begin{tabular}{ll}
        a) $2 \cdot 2 \cdot 2 \cdot 2$ = $2^{\phantom{0}}$ = \underline{\hspace{3cm}} & b) $5 \cdot 5 \cdot 5$ = $5^{\phantom{0}}$ = \underline{\hspace{3cm}} \\[3ex]
        c) $3 \cdot 3 \cdot 3 \cdot 3 \cdot 3$ = $3^{\phantom{0}}$ = \underline{\hspace{3cm}} & d) $7 \cdot 7$ = $7^{\phantom{0}}$ = \underline{\hspace{3cm}}
    \end{tabular}

    \vspace{1cm}

    \item \textbf{Berechne die Potenzwerte:}
    \vspace{0.5cm}

    \begin{tabular}{ll}
        a) $2^3$ = $2 \cdot 2 \cdot 2$ = \underline{\hspace{3cm}} & b) $3^2$ = $3 \cdot 3$ = \underline{\hspace{3cm}} \\[3ex]
        c) $4^2$ = $4 \cdot 4$ = \underline{\hspace{3cm}} & d) $5^2$ = $5 \cdot 5$ = \underline{\hspace{3cm}} \\[3ex]
        e) $2^4$ = $2 \cdot 2 \cdot 2 \cdot 2$ = \underline{\hspace{3cm}} & f) $10^3$ = $10 \cdot 10 \cdot 10$ = \underline{\hspace{3cm}}
    \end{tabular}

    \vspace{1cm}

    \item \textbf{Negative Exponenten - Schreibe als Bruch:}
    \textit{Beispiel:} $2^{-1} = \dfrac{1}{2^1} = \dfrac{1}{2}$
    \vspace{0.5cm}

    \begin{enumerate}[label=\alph*)]
        \item $3^{-1}$ = $\dfrac{1}{3^{\phantom{0}}}$ = $\dfrac{\phantom{0}}{\phantom{0}}$ = \underline{\hspace{3cm}}
        \vspace{0.5cm}

        \item $2^{-2}$ = $\dfrac{1}{2^{\phantom{0}}}$ = $\dfrac{1}{\phantom{00}}$ = \underline{\hspace{3cm}}
        \vspace{0.5cm}

        \item $5^{-1}$ = $\dfrac{1}{5^{\phantom{0}}}$ = $\dfrac{\phantom{0}}{\phantom{0}}$ = \underline{\hspace{3cm}}
    \end{enumerate}

\end{enumerate}