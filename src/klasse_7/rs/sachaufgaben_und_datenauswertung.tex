\subsection*{Teil D: Sachaufgaben und Datenauswertung (20 Minuten)}

\begin{enumerate}[label=\arabic*.]

    \item \textbf{Zahlenrätsel:}
    \textit{Denke dir eine Zahl. Multipliziere sie mit 4 und subtrahiere 7. Das Ergebnis ist 21. Wie heißt die gedachte Zahl?}
    \vspace{0.5cm}

    Gleichung: $4x - 7 = 21$ \\
    Lösung: $x = \phantom{00}$

    \vspace{1cm}

    \item \textbf{Altersaufgabe:}
    \textit{Ein Vater ist dreimal so alt wie sein Sohn. In 12 Jahren wird er nur noch doppelt so alt sein. Wie alt sind beide heute?}
    \vspace{0.5cm}

    Heute: Sohn $x$ Jahre, Vater $3x$ Jahre \\
    In 12 Jahren: Sohn $(x + 12)$ Jahre, Vater $(3x + 12)$ Jahre \\
    Gleichung: $3x + 12 = 2(x + 12)$ \\
    $3x + 12 = 2x + 24$ \\
    $x = \phantom{00}$ \\
    Antwort: Sohn \underline{\hspace{1cm}} Jahre, Vater \underline{\hspace{1cm}} Jahre

    \vspace{1cm}

    \item \textbf{Datenauswertung:}
    \textit{Notendurchschnitte von 6 Schülern: 2,1 - 2,7 - 1,8 - 3,2 - 2,5 - 2,2}
    \vspace{0.5cm}

    \begin{enumerate}[label=\alph*)]
        \item Geordnet: \underline{\hspace{6cm}}
        \vspace{0.3cm}

        \item Arithmetisches Mittel: $\overline{x} = \dfrac{\phantom{00}}{6} = \phantom{0,0}$
        \vspace{0.3cm}

        \item Zentralwert (Median): \underline{\hspace{2cm}}
        \vspace{0.3cm}

        \item Spannweite: $\phantom{0,0} - \phantom{0,0} = \phantom{0,0}$
    \end{enumerate}

\end{enumerate}