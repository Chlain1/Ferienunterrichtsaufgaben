\subsection*{Teil C: Ungleichungen und Sachaufgaben (25 Minuten)}

\begin{enumerate}[label=\arabic*.]

    \item \textbf{Lineare Ungleichungen lösen:}
    \textit{Beispiel:} $2x + 3 < 11$ $|-3$ $\Rightarrow$ $2x < 8$ $|:2$ $\Rightarrow$ $x < 4$
    \vspace{0.5cm}

    \begin{enumerate}[label=\alph*)]
        \item $x + 5 > 12$ \\
        $x + 5 > 12$ $|\phantom{00}$ \\
        $x > \phantom{00}$ \\
        Lösungsmenge: $L = \{\phantom{000000000000}\}$
        \vspace{0.5cm}

        \item $3x - 2 \leq 10$ \\
        $3x - 2 \leq 10$ $|\phantom{00}$ \\
        $3x \leq \phantom{00}$ $|\phantom{00}$ \\
        $x \leq \phantom{00}$ \\
        Lösungsmenge: $L = \{\phantom{000000000000}\}$
        \vspace{0.5cm}

        \item $2x + 1 \geq 9$ \\
        $2x + 1 \geq 9$ $|\phantom{00}$ \\
        $2x \geq \phantom{00}$ $|\phantom{00}$ \\
        $x \geq \phantom{00}$ \\
        Lösungsmenge: $L = \{\phantom{000000000000}\}$
    \end{enumerate}

    \vspace{1cm}

    \item \textbf{Sachaufgaben mit Gleichungen:}
    \vspace{0.5cm}

    \begin{enumerate}[label=\alph*)]
        \item \textbf{Zahlenrätsel:} Das Dreifache einer Zahl vermehrt um 5 ergibt 23. Wie heißt die Zahl? \\
        Gleichung: $3x + 5 = 23$ \\
        Lösung: $x = \phantom{00}$
        \vspace{0.5cm}

        \item \textbf{Alter:} Peter ist 3 Jahre älter als sein Bruder. Zusammen sind sie 25 Jahre alt. Wie alt ist Peter? \\
        Bruder: $x$ Jahre, Peter: $x + 3$ Jahre \\
        Gleichung: $x + (x + 3) = 25$ \\
        Vereinfacht: $2x + 3 = 25$ \\
        Lösung: Bruder $= \phantom{00}$ Jahre, Peter $= \phantom{00}$ Jahre
        \vspace{0.5cm}

        \item \textbf{Geld:} Anna hat 8€ mehr als Ben. Zusammen haben sie 32€. Wie viel Geld hat jeder? \\
        Ben: $x$ €, Anna: $x + 8$ € \\
        Gleichung: $\phantom{00000000000000}$ \\
        Lösung: Ben $= \phantom{00}$ €, Anna $= \phantom{00}$ €
    \end{enumerate}

    \vspace{1cm}

    \item \textbf{Begriffe mit Ungleichungen:}
    \textit{Übersetze in mathematische Sprache:}
    \vspace{0.5cm}

    \begin{enumerate}[label=\alph*)]
        \item "höchstens 10": $x \phantom{0} 10$
        \vspace{0.3cm}

        \item "mindestens 5": $x \phantom{0} 5$
        \vspace{0.3cm}

        \item "weniger als 8": $x \phantom{0} 8$
        \vspace{0.3cm}

        \item "mehr als 12": $x \phantom{0} 12$
    \end{enumerate}

\end{enumerate}