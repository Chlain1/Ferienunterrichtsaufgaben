\subsection*{Teil C: Zehnerpotenzen und Einheiten (35 Minuten)}

\begin{enumerate}[label=\arabic*.]

    \item \textbf{Schreibe als Zehnerpotenz:}
    \vspace{0.5cm}

    \begin{tabular}{ll}
        a) 100 = $10^{\phantom{0}}$ & b) 1000 = $10^{\phantom{0}}$ \\[2ex]
        c) 10000 = $10^{\phantom{0}}$ & d) 0,1 = $10^{\phantom{0}}$ \\[2ex]
        e) 0,01 = $10^{\phantom{0}}$ & f) 0,001 = $10^{\phantom{0}}$
    \end{tabular}

    \vspace{1cm}

    \item \textbf{Wandle um mit Zehnerpotenzen:}
    \textit{Beispiel:} 2 km = 2000 m = $2 \cdot 10^3$ m
    \vspace{0.5cm}

    \begin{enumerate}[label=\alph*)]
        \item 3 km = \underline{\hspace{2cm}} m = $3 \cdot 10^{\phantom{0}}$ m
        \vspace{0.5cm}

        \item 5 mm = \underline{\hspace{2cm}} m = $5 \cdot 10^{\phantom{0}}$ m
        \vspace{0.5cm}

        \item 4 cm = \underline{\hspace{2cm}} m = $4 \cdot 10^{\phantom{0}}$ m
        \vspace{0.5cm}

        \item 7 g = \underline{\hspace{2cm}} kg = $7 \cdot 10^{\phantom{0}}$ kg
    \end{enumerate}

    \vspace{1cm}

    \item \textbf{Gemischte Aufgaben:}
    \vspace{0.5cm}

    \begin{enumerate}[label=\alph*)]
        \item $2^3 \cdot 2^{-1}$ = $2^{\phantom{0}+(\phantom{0})}$ = $2^{\phantom{0}}$ = \underline{\hspace{3cm}}
        \vspace{0.5cm}

        \item $10^2 \cdot 10^{-3}$ = $10^{\phantom{0}+(\phantom{0})}$ = $10^{\phantom{0}}$ = \underline{\hspace{3cm}}
        \vspace{0.5cm}

        \item $(3^{-1})^2$ = $3^{(\phantom{0}) \cdot \phantom{0}}$ = $3^{\phantom{0}}$ = \underline{\hspace{3cm}}
    \end{enumerate}

\end{enumerate}

\vspace{1cm}
\textbf{Hinweise für Schüler:}
\begin{itemize}
    \item Bei negativen Exponenten: $a^{-n} = \dfrac{1}{a^n}$
    \item $a^0 = 1$ für alle $a \neq 0$
    \item Zehnerpotenzen helfen beim Umrechnen von Einheiten
\end{itemize}