\subsection*{Teil D: Terme interpretieren und vergleichen (20 Minuten)}

\begin{enumerate}[label=\arabic*., resume]

    \item \textbf{Erkläre in Worten, was die Terme bedeuten:}

    \vspace{0.5cm}

    \begin{enumerate}[label=\alph*)]
        \item $x + 8$: \underline{\hspace{8cm}}
        \vspace{0.5cm}
        \item $5x$: \underline{\hspace{8cm}}
        \vspace{0.5cm}
        \item $20 - x$: \underline{\hspace{8cm}}
        \vspace{0.5cm}
        \item $x \div 2$: \underline{\hspace{8cm}}
    \end{enumerate}

    \vspace{1cm}

    \item \textbf{Welcher Term ist größer? Setze $<$, $>$ oder $=$ ein.} \\
    (Berechne für $x = 6$)

    \vspace{0.5cm}

    \begin{tabular}{ll}
        a) $2x$ \phantom{$=$} $x + 6$ & (Rechnung: \underline{\hspace{4cm}}) \\[2ex]
        b) $3x - 4$ \phantom{$=$} $2x + 8$ & (Rechnung: \underline{\hspace{4cm}}) \\[2ex]
        c) $x + x + x$ \phantom{$=$} $3x$ & (Rechnung: \underline{\hspace{4cm}})
    \end{tabular}

    \vspace{1cm}

    \item \textbf{Bonusaufgabe:}

    Familie Müller geht ins Restaurant. Das Menü kostet x Euro pro Person. Die Familie besteht aus 4 Personen. Zusätzlich bestellen sie Getränke für insgesamt 12 Euro und geben 8 Euro Trinkgeld.

    \vspace{0.5cm}

    \begin{enumerate}[label=\alph*)]
        \item Stelle einen Term für die Gesamtkosten auf: \underline{\hspace{4cm}}
        \vspace{0.5cm}
        \item Berechne die Kosten, wenn ein Menü 18 Euro kostet: \underline{\hspace{4cm}}
    \end{enumerate}

\end{enumerate}