\subsection*{Teil A: Einfache Gleichungen (25 Minuten)}

\begin{merkbox}[Das Waage-Prinzip]
    Eine Gleichung ist wie eine Waage im Gleichgewicht. Was du auf einer Seite machst, musst du auch auf der anderen Seite machen!\\
    \textbf{Äquivalenzumformungen:} $+a$, $-a$, $\cdot a$, $\div a$ (mit $a \neq 0$)
\end{merkbox}

\begin{enumerate}[label=\arabic*.]

    \item \textbf{Löse die Gleichungen durch Äquivalenzumformungen:}

    \vspace{0.5cm}

    \begin{tabular}{l}
        a) $x + 7 = 15$ \\[2ex]
        $x + 7 \color{red}{- 7} = 15 \color{red}{- 7}$ \\[1ex]
        $x = \phantom{00}$ = \underline{\hspace{2cm}} \\[3ex]
        \textbf{Probe:} $\phantom{0} + 7 = $ \underline{\hspace{2cm}} $\checkmark$ \\[4ex]

        b) $x - 4 = 11$ \\[2ex]
        $x - 4 \color{red}{+ 4} = 11 \color{red}{+ 4}$ \\[1ex]  
        $x = $ \underline{\hspace{2cm}} \\[3ex]
        \textbf{Probe:} \underline{\hspace{4cm}} \\[4ex]

        c) $3x = 21$ \\[2ex]
        $3x \color{red}{\div 3} = 21 \color{red}{\div 3}$ \\[1ex]
        $x = $ \underline{\hspace{2cm}} \\[3ex]
        \textbf{Probe:} \underline{\hspace{4cm}} \\[4ex]

        d) $\dfrac{x}{4} = 5$ \\[2ex]
        $\dfrac{x}{4} \color{red}{\cdot 4} = 5 \color{red}{\cdot 4}$ \\[1ex]
        $x = $ \underline{\hspace{2cm}} \\[3ex]
        \textbf{Probe:} \underline{\hspace{4cm}}
    \end{tabular}

    \vspace{1cm}

    \item \textbf{Löse selbstständig:}

    \vspace{0.5cm}

    \begin{tabular}{ll}
        a) $x + 9 = 23$ & $x = $ \underline{\hspace{2cm}} \\[3ex]
        b) $x - 6 = 18$ & $x = $ \underline{\hspace{2cm}} \\[3ex]
        c) $5x = 35$ & $x = $ \underline{\hspace{2cm}} \\[3ex]
        d) $\dfrac{x}{7} = 3$ & $x = $ \underline{\hspace{2cm}}
    \end{tabular}

    \vspace{1cm}

    \item \textbf{Gleichungen mit zwei Schritten:}

    \vspace{0.5cm}

    \begin{beispielbox}[Beispiel]
        $2x + 3 = 11$ \\
        $2x + 3 - 3 = 11 - 3$ \quad (erst $-3$)\\
        $2x = 8$ \\
        $2x \div 2 = 8 \div 2$ \quad (dann $\div 2$)\\
        $x = 4$
    \end{beispielbox}

    \vspace{0.5cm}

    \begin{enumerate}[label=\alph*)]
        \item $3x + 5 = 20$

        \vspace{0.3cm}
        $3x + 5 - \phantom{0} = 20 - \phantom{0}$
        \vspace{0.3cm}
        $3x = \phantom{00}$
        \vspace{0.3cm}
        $x = $ \underline{\hspace{2cm}}

        \vspace{0.5cm}

        \item $4x - 7 = 13$

        \vspace{0.3cm}
        $4x - 7 + \phantom{0} = 13 + \phantom{0}$
        \vspace{0.3cm}
        $4x = \phantom{00}$
        \vspace{0.3cm}
        $x = $ \underline{\hspace{2cm}}

    \end{enumerate}

\end{enumerate}