\subsection*{Teil C: Prozentrechnung (25 Minuten)}

\begin{merkbox}[Prozentformeln]
    \textbf{Prozentwert (W):} $W = \dfrac{p \cdot G}{100}$ \\
    \textbf{Prozentsatz (p):} $p = \dfrac{W \cdot 100}{G}$ \\  
    \textbf{Grundwert (G):} $G = \dfrac{W \cdot 100}{p}$ \\
    G = Grundwert, W = Prozentwert, p = Prozentsatz
\end{merkbox}

\begin{enumerate}[label=\arabic*., resume]

    \item \textbf{Berechne den Prozentwert:}

    \vspace{0.5cm}

    \begin{tabular}{l}
        a) 15\% von 200 Euro

        \vspace{0.3cm}
        $W = \dfrac{15 \cdot 200}{100} = \dfrac{\phantom{0000}}{100} = $ \underline{\hspace{2cm}} Euro

        \vspace{0.5cm}

        b) 8\% von 150 Euro

        \vspace{0.3cm}
        $W = \dfrac{\phantom{0} \cdot \phantom{000}}{100} = $ \underline{\hspace{2cm}} Euro

        \vspace{0.5cm}

        c) 25\% von 80 Euro

        \vspace{0.3cm}
        $W = $ \underline{\hspace{4cm}} Euro
    \end{tabular}

    \vspace{1cm}

    \item \textbf{Berechne den Prozentsatz:}

    \vspace{0.5cm}

    \begin{enumerate}[label=\alph*)]
        \item 12 von 60 Schülern tragen eine Brille. Wie viel Prozent sind das?

        \vspace{0.3cm}
        $p = \dfrac{12 \cdot 100}{60} = \dfrac{\phantom{0000}}{60} = $ \underline{\hspace{2cm}} \%

        \vspace{0.5cm}

        \item Ein T-Shirt kostet 25 Euro, im Sale kostet es 20 Euro. Um wie viel Prozent wurde es reduziert?

        \vspace{0.3cm}
        Rabatt: $25 - 20 = $ \underline{\hspace{2cm}} Euro
        \vspace{0.3cm}
        $p = \dfrac{\phantom{0} \cdot 100}{25} = $ \underline{\hspace{2cm}} \%

    \end{enumerate}

    \vspace{1cm}

    \item \textbf{Berechne den Grundwert:}

    \vspace{0.5cm}

    \begin{enumerate}[label=\alph*)]
        \item 20\% entsprechen 15 Euro. Wie viel sind 100\%?

        \vspace{0.3cm}
        $G = \dfrac{15 \cdot 100}{20} = $ \underline{\hspace{2cm}} Euro

        \vspace{0.5cm}

        \item In einer Klasse sind 6 Schüler krank. Das sind 25\% der Klasse. Wie viele Schüler sind in der Klasse?

        \vspace{0.3cm}
        $G = \dfrac{\phantom{0} \cdot 100}{\phantom{00}} = $ \underline{\hspace{2cm}} Schüler

    \end{enumerate}

    \vspace{1cm}

    \item \textbf{Gemischte Aufgaben:}

    \vspace{0.5cm}

    \begin{center}
        \begin{tabular}{|l|c|c|c|}
            \hline
            & Grundwert (G) & Prozentsatz (p) & Prozentwert (W) \\
            \hline
            a) & 400 Euro & 12\% & \phantom{00} Euro \\
            \hline
            b) & 150 Stück & \phantom{00}\% & 30 Stück \\
            \hline
            c) & \phantom{000} kg & 8\% & 6 kg \\
            \hline
            d) & 80 Personen & 35\% & \phantom{00} Personen \\
            \hline
        \end{tabular}
    \end{center}

\end{enumerate}