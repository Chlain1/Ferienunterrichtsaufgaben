\subsection*{Teil B: Gleichungen mit Klammern (20 Minuten)}

\begin{warnbox}[Achtung!]
    Erst Klammern auflösen, dann gleichnamige Terme zusammenfassen, dann lösen!
\end{warnbox}

\begin{enumerate}[label=\arabic*., resume]

    \item \textbf{Löse die Gleichungen mit Klammern:}

    \vspace{0.5cm}

    \begin{enumerate}[label=\alph*)]
        \item $2(x + 3) = 14$

        \vspace{0.3cm}
        $2x + \phantom{0} = 14$ \quad (Klammer auflösen)
        \vspace{0.3cm}
        $2x + 6 - \phantom{0} = 14 - \phantom{0}$ \quad ($-6$)
        \vspace{0.3cm}
        $2x = \phantom{0}$
        \vspace{0.3cm}
        $x = $ \underline{\hspace{2cm}}

        \vspace{1cm}

        \item $3(x - 2) = 15$

        \vspace{0.3cm}
        $3x - \phantom{0} = 15$ \quad (Klammer auflösen)
        \vspace{0.3cm}
        $3x - 6 + \phantom{0} = 15 + \phantom{0}$ \quad ($+6$)
        \vspace{0.3cm}
        $3x = \phantom{00}$
        \vspace{0.3cm}
        $x = $ \underline{\hspace{2cm}}

        \vspace{1cm}

        \item $4(x + 1) - 2x = 18$

        \vspace{0.3cm}
        $4x + \phantom{0} - 2x = 18$ \quad (Klammer auflösen)
        \vspace{0.3cm}
        $\phantom{0}x + 4 = 18$ \quad (zusammenfassen)
        \vspace{0.3cm}
        $2x = \phantom{00}$ \quad ($-4$)
        \vspace{0.3cm}
        $x = $ \underline{\hspace{2cm}}

    \end{enumerate}

    \vspace{1cm}

    \item \textbf{Textgleichungen:}

    \vspace{0.5cm}

    \begin{enumerate}[label=\alph*)]
        \item Das Dreifache einer Zahl vermehrt um 7 ergibt 25. Wie heißt die Zahl?

        \vspace{0.3cm}
        Gleichung: $3x + 7 = \phantom{00}$
        \vspace{0.3cm}
        Lösung: $x = $ \underline{\hspace{2cm}}

        \vspace{0.5cm}

        \item Wenn ich zu einer Zahl 12 addiere und das Ergebnis mit 2 multipliziere, erhalte ich 30.

        \vspace{0.3cm}
        Gleichung: $2(x + \phantom{00}) = \phantom{00}$
        \vspace{0.3cm}
        Lösung: $x = $ \underline{\hspace{2cm}}

    \end{enumerate}

\end{enumerate}