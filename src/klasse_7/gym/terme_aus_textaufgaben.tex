\subsection*{Teil B: Terme aus Textaufgaben aufstellen (25 Minuten)}

\begin{beispielbox}[Beispiel]
"Lisa kauft 3 Kugeln Eis. Eine Kugel kostet x Euro."\\
Term für die Gesamtkosten: $3 \cdot x$ oder kurz $3x$
\end{beispielbox}

\begin{enumerate}[label=\arabic*., resume]

    \item \textbf{Stelle zu jeder Situation einen Term auf:}

    \vspace{0.5cm}

    \begin{enumerate}[label=\alph*)]
        \item Tim ist x Jahre alt. Wie alt ist er in 5 Jahren?
        \vspace{0.3cm}

        Term: \underline{\hspace{4cm}}

        \vspace{0.5cm}

        \item Ein Handy kostet y Euro. Es gibt 20 Euro Rabatt. Wie viel kostet es dann?
        \vspace{0.3cm}

        Term: \underline{\hspace{4cm}}

        \vspace{0.5cm}

        \item In einem Kino sind 150 Plätze. x Plätze sind besetzt. Wie viele Plätze sind noch frei?
        \vspace{0.3cm}

        Term: \underline{\hspace{4cm}}

        \vspace{0.5cm}

        \item Sarah hat a Euro Taschengeld. Sie bekommt von ihrer Oma noch 10 Euro dazu und kauft sich für 15 Euro ein Buch. Wie viel Geld hat sie noch?
        \vspace{0.3cm}

        Term: \underline{\hspace{4cm}}

    \end{enumerate}

    \vspace{1cm}

    \item \textbf{Berechne für die Terme aus Aufgabe 4 die konkreten Werte:}

    \vspace{0.5cm}

    \begin{enumerate}[label=\alph*)]
        \item Für $x = 12$: \underline{\hspace{3cm}}
        \vspace{0.3cm}
        \item Für $y = 300$: \underline{\hspace{3cm}}
        \vspace{0.3cm}
        \item Für $x = 80$: \underline{\hspace{3cm}}
        \vspace{0.3cm}
        \item Für $a = 25$: \underline{\hspace{3cm}}
    \end{enumerate}

\end{enumerate}
