\subsection*{Teil D: Realitätsbezogene Textaufgaben (20 Minuten)}

\begin{enumerate}[label=\arabic*., resume]

    \item \textbf{Handyvertrag:}

    Ein Handyvertrag kostet 25 Euro Grundgebühr plus 0,15 Euro pro SMS.

    \vspace{0.5cm}

    \begin{enumerate}[label=\alph*)]
        \item Stelle eine Gleichung für die monatlichen Kosten auf, wenn x SMS versendet werden:

        \vspace{0.3cm}
        Kosten = \underline{\hspace{6cm}}

        \vspace{0.5cm}

        \item Lisa zahlt im Monat 31 Euro. Wie viele SMS hat sie versendet?

        \vspace{0.3cm}
        Gleichung: $25 + 0{,}15 \cdot x = 31$
        \vspace{0.3cm}
        $0{,}15x = 31 - \phantom{00}$
        \vspace{0.3cm}
        $0{,}15x = \phantom{0}$
        \vspace{0.3cm}
        $x = $ \underline{\hspace{2cm}} SMS

    \end{enumerate}

    \vspace{1cm}

    \item \textbf{Rabattaktion:}

    In einem Sportgeschäft gibt es 20\% Rabatt auf alle Artikel.

    \vspace{0.5cm}

    \begin{enumerate}[label=\alph*)]
        \item Ein Fußball kostete ursprünglich 35 Euro. Wie viel kostet er jetzt?

        \vspace{0.3cm}
        Rabatt: $\dfrac{20 \cdot 35}{100} = $ \underline{\hspace{2cm}} Euro
        \vspace{0.3cm}
        Neuer Preis: $35 - \phantom{0} = $ \underline{\hspace{2cm}} Euro

        \vspace{0.5cm}

        \item Tim kauft Sportschuhe und zahlt nach Rabatt 72 Euro. Wie viel kosteten sie ursprünglich?

        \vspace{0.3cm}
        \textit{Überlegung:} 72 Euro entsprechen 80\% des ursprünglichen Preises.
        \vspace{0.3cm}
        Gleichung: $\dfrac{80 \cdot G}{100} = 72$
        \vspace{0.3cm}
        $G = \dfrac{72 \cdot 100}{80} = $ \underline{\hspace{2cm}} Euro

    \end{enumerate}

    \vspace{1cm}

    \item \textbf{Zinsrechnung:}

    Sarah legt 500 Euro zu 3\% Zinsen pro Jahr an.

    \vspace{0.5cm}

    \begin{enumerate}[label=\alph*)]
        \item Wie viele Zinsen erhält sie nach einem Jahr?

        \vspace{0.3cm}
        Zinsen = $\dfrac{\phantom{0} \cdot \phantom{000}}{100} = $ \underline{\hspace{2cm}} Euro

        \vspace{0.5cm}

        \item Wie viel Geld hat sie nach einem Jahr insgesamt?

        \vspace{0.3cm}
        Gesamtbetrag = $500 + \phantom{00} = $ \underline{\hspace{2cm}} Euro

    \end{enumerate}

    \vspace{1cm}

    \item \textbf{Klassenarbeit-Analyse:}

    In einer Mathematikarbeit haben von 28 Schülern 21 eine gute Note (Note 1 oder 2) erhalten.

    \vspace{0.5cm}

    \begin{enumerate}[label=\alph*)]
        \item Wie viel Prozent der Schüler haben eine gute Note erhalten?

        \vspace{0.3cm}
        $p = \dfrac{21 \cdot 100}{28} = $ \underline{\hspace{2cm}} \%

        \vspace{0.5cm}

        \item Der Lehrer möchte, dass mindestens 80\% der Schüler eine gute Note haben. Wie viele Schüler müssten das mindestens sein?

        \vspace{0.3cm}
        $W = \dfrac{80 \cdot 28}{100} = $ \underline{\hspace{2cm}} Schüler

        \vspace{0.5cm}

        \item Wurde das Ziel erreicht? \underline{\hspace{3cm}}

    \end{enumerate}

    \vspace{1cm}

    \item \textbf{Komplexe Anwendungsaufgabe:}

    Familie Weber plant den Kauf eines Autos für 18.000 Euro. Sie zahlen 30\% sofort an und finanzieren den Rest über einen Kredit mit 4\% Zinsen pro Jahr.

    \vspace{0.5cm}

    \begin{enumerate}[label=\alph*)]
        \item Wie viel Euro zahlen sie sofort?

        \vspace{0.3cm}
        Anzahlung = $\dfrac{\phantom{00} \cdot \phantom{00000}}{100} = $ \underline{\hspace{3cm}} Euro

        \vspace{0.5cm}

        \item Wie hoch ist die Kreditsumme?

        \vspace{0.3cm}
        Kredit = $18000 - \phantom{0000} = $ \underline{\hspace{3cm}} Euro

        \vspace{0.5cm}

        \item Wie viele Zinsen zahlen sie im ersten Jahr für den Kredit?

        \vspace{0.3cm}
        Zinsen = $\dfrac{4 \cdot \phantom{00000}}{100} = $ \underline{\hspace{3cm}} Euro

    \end{enumerate}

\end{enumerate}