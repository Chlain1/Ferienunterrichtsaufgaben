\subsection*{Teil A: Terme und Variablen (20 Minuten)}
\textit{(Wiederholung aus Arbeitsblatt 1)}

\begin{enumerate}[label=\arabic*.]

    \item \textbf{Berechne die Terme für $x = 6$:}

    \vspace{0.5cm}

    \begin{tabular}{ll}
        a) $3x + 8 = 3 \cdot \phantom{0} + 8 = $ \underline{\hspace{2cm}} & b) $5x - 12 = $ \underline{\hspace{3cm}} \\[3ex]
        c) $2(x + 4) = $ \underline{\hspace{3cm}} & d) $x^2 - 10 = $ \underline{\hspace{3cm}}
    \end{tabular}

    \vspace{1cm}

    \item \textbf{Vereinfache die Terme (verwende Potenz- und Distributivgesetz):}

    \vspace{0.5cm}

    \begin{enumerate}[label=\alph*)]
        \item $4(x + 3) - 2(x - 1) = 4x + \phantom{00} - 2x + \phantom{0} = \phantom{0}x + \phantom{00}$ = \underline{\hspace{3cm}}

        \vspace{0.5cm}

        \item $x^3 \cdot x^4 + 2x^2 \cdot x^5 = x^{\phantom{0}} + 2x^{\phantom{0}} = x^{\phantom{0}} + \phantom{0}x^{\phantom{0}}$ = \underline{\hspace{3cm}}

        \vspace{0.5cm}

        \item $(x + 2)^2 = x^2 + 2 \cdot x \cdot \phantom{0} + \phantom{0}^2 = x^2 + \phantom{0}x + \phantom{0}$ = \underline{\hspace{3cm}}

    \end{enumerate}

    \vspace{1cm}

    \item \textbf{Textaufgabe:}

    Maria kauft $x$ Hefte zu je 2,50 Euro und 3 Stifte zu je 1,20 Euro.

    \vspace{0.5cm}

    \begin{enumerate}[label=\alph*)]
        \item Stelle einen Term für die Gesamtkosten auf: \underline{\hspace{4cm}}

        \vspace{0.5cm}

        \item Berechne die Kosten für 4 Hefte: \underline{\hspace{4cm}}

    \end{enumerate}

\end{enumerate}