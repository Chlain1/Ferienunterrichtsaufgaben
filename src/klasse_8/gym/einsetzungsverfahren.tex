\subsection*{Teil B: Einsetzungsverfahren (30 Minuten)}

\begin{enumerate}[label=\arabic*., resume]

    \item \textbf{Löse mit dem Einsetzungsverfahren:}
    \vspace{0.5cm}

    \textit{Beispiel:} $\begin{cases} y = 2x + 1 \\ x + y = 4 \end{cases}$

    Setze erste Gleichung in die zweite ein: $x + (2x + 1) = 4$
    $3x + 1 = 4$, also $x = 1$ und $y = 2 \cdot 1 + 1 = 3$

    \begin{enumerate}[label=\alph*)]
        \item $\begin{cases} y = x + 2 \\ 2x + y = 8 \end{cases}$

        Einsetzen: $2x + (x + 2) = 8$

        $3x + 2 = 8$ $\Rightarrow$ $3x =$ \underline{\hspace{1cm}} $\Rightarrow$ $x =$ \underline{\hspace{1cm}}

        $y = 2 + 2 =$ \underline{\hspace{1cm}}

        Lösung: $x =$ \underline{\hspace{1cm}}, $y =$ \underline{\hspace{1cm}}

        \vspace{0.5cm}
        \item $\begin{cases} x = y - 1 \\ 3x + 2y = 7 \end{cases}$

        Einsetzen: $3(y - 1) + 2y = 7$

        \underline{\hspace{8cm}} $\Rightarrow$ $y =$ \underline{\hspace{1cm}}

        $x = 2 - 1 =$ \underline{\hspace{1cm}}

        Lösung: $x =$ \underline{\hspace{1cm}}, $y =$ \underline{\hspace{1cm}}

        \vspace{0.5cm}
        \item $\begin{cases} 2x + y = 5 \\ x - y = 1 \end{cases}$

        Aus der zweiten Gleichung: $x = y + 1$

        Einsetzen: \underline{\hspace{8cm}}

        Lösung: $x =$ \underline{\hspace{1cm}}, $y =$ \underline{\hspace{1cm}}
    \end{enumerate}

    \vspace{1cm}

    \item \textbf{Forme erst um, dann löse:}
    \vspace{0.5cm}

    $\begin{cases} 2x + 3y = 11 \\ x - y = 1 \end{cases}$

    Aus der zweiten Gleichung: $x =$ \underline{\hspace{2cm}}

    Einsetzen in die erste: $2($ \underline{\hspace{1.5cm}} $) + 3y = 11$

    \underline{\hspace{8cm}}

    Lösung: $x =$ \underline{\hspace{1cm}}, $y =$ \underline{\hspace{1cm}}

\end{enumerate}