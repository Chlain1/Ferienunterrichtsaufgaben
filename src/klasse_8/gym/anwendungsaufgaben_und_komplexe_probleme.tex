\subsection*{Teil C: Anwendungsaufgaben und komplexe Probleme (30 Minuten)}

\begin{enumerate}[label=\arabic*., resume]

    \item \textbf{Handy-Tarif (Funktionen):}
    \vspace{0.5cm}

    Tarif A: 15€ Grundgebühr + 0,10€/min, Tarif B: 5€ Grundgebühr + 0,25€/min

    \begin{enumerate}[label=\alph*)]
        \item Funktionsgleichungen: $A(x) =$ \underline{\hspace{3cm}}, $B(x) =$ \underline{\hspace{3cm}}
        \vspace{0.3cm}
        \item Ab wie vielen Minuten ist Tarif A günstiger?

        $15 + 0{,}1x < 5 + 0{,}25x$ $\Rightarrow$ $10 < 0{,}15x$ $\Rightarrow$ $x >$ \underline{\hspace{2cm}} Minuten
    \end{enumerate}

    \vspace{1cm}

    \item \textbf{Zahlenrätsel (Gleichungssystem):}
    \vspace{0.5cm}

    Zwei Zahlen haben die Summe 50. Wenn man die erste durch die zweite teilt, erhält man $\dfrac{3}{2}$.

    Seien die Zahlen $x$ und $y$ mit $x + y = 50$ und $\dfrac{x}{y} = \dfrac{3}{2}$.

    Aus der zweiten Gleichung: $x = \dfrac{3y}{2}$

    Einsetzen: $\dfrac{3y}{2} + y = 50$ $\Rightarrow$ $\dfrac{5y}{2} = 50$ $\Rightarrow$ $y =$ \underline{\hspace{2cm}}, $x =$ \underline{\hspace{2cm}}

    \vspace{1cm}

    \item \textbf{Wassertank (Zylinder):}
    \vspace{0.5cm}

    Ein zylindrischer Wassertank mit Radius 80 cm und Höhe 2{,}5 m soll gestrichen werden.

    \begin{enumerate}[label=\alph*)]
        \item Wie groß ist die zu streichende Oberfläche (ohne Boden)?

        Mantelfläche + Deckel: $M + G = 2\pi rh + \pi r^2$

        $r = 0{,}8$ m, $h = 2{,}5$ m

        $O = 2\pi \cdot 0{,}8 \cdot 2{,}5 + \pi \cdot 0{,}8^2 = 4\pi + 0{,}64\pi = 4{,}64\pi \approx$ \underline{\hspace{2cm}} m$^2$

        \vspace{0.3cm}
        \item Wie viele Liter Wasser passen in den Tank?

        $V = \pi r^2 h = \pi \cdot 0{,}64 \cdot 2{,}5 = 1{,}6\pi \approx$ \underline{\hspace{2cm}} m$^3$ = \underline{\hspace{2cm}} Liter
    \end{enumerate}

    \vspace{1cm}

    \item \textbf{Komplexaufgabe - Kreissegment und Funktion:}
    \vspace{0.5cm}

    Ein Halbkreis mit Radius $r = 5$ cm wird durch eine Sehne geteilt. Die Sehne ist 3 cm vom Mittelpunkt entfernt.

    \begin{enumerate}[label=\alph*)]
        \item Die Fläche des gesamten Halbkreises beträgt: $A = \dfrac{1}{2}\pi r^2 = \dfrac{25\pi}{2} \approx$ \underline{\hspace{2cm}} cm$^2$

        \vspace{0.3cm}
        \item Ein Rechteck soll in einen Kreis mit Radius 4 cm einbeschrieben werden. Wenn eine Seite $x$ cm lang ist, dann gilt für die andere Seite $y$: $x^2 + y^2 = 64$

        Stelle $y$ in Abhängigkeit von $x$ dar: $y = \sqrt{64 - x^2}$

        Die Rechteckfläche als Funktion: $A(x) = x \cdot \sqrt{64 - x^2}$

        Für $x = 4$ cm ist $y =$ \underline{\hspace{2cm}} cm und $A(4) =$ \underline{\hspace{2cm}} cm$^2$
    \end{enumerate}

    \vspace{1cm}

    \item \textbf{Abschlussprobe - Alles kombiniert:}
    \vspace{0.5cm}

    Gegeben ist die Funktion $f(x) = \dfrac{2x + 4}{x - 1}$ und das Gleichungssystem $\begin{cases} x + 2y = 8 \\ 2x - y = 1 \end{cases}$

    \begin{enumerate}[label=\alph*)]
        \item Definitionsmenge von $f$: $\mathbb{D} =$ \underline{\hspace{3cm}}
        \vspace{0.3cm}
        \item Lösung des Gleichungssystems: $x =$ \underline{\hspace{1.5cm}}, $y =$ \underline{\hspace{1.5cm}}
        \vspace{0.3cm}
        \item Berechne $f(2) =$ \underline{\hspace{2cm}}
    \end{enumerate}

\end{enumerate}
