\subsection*{Teil C: Anwendungsaufgaben und komplexere Berechnungen (30 Minuten)}

\begin{enumerate}[label=\arabic*., resume]

    \item \textbf{Umgekehrte Berechnungen:}
    \vspace{0.5cm}

    \begin{enumerate}[label=\alph*)]
        \item Ein Zylinder hat das Volumen $V = 200\pi$ cm$^3$ und die Höhe $h = 8$ cm. Berechne den Radius.

        $V = \pi r^2 h$ $\Rightarrow$ $200\pi = \pi r^2 \cdot 8$ $\Rightarrow$ $r^2 = \dfrac{200}{8} = 25$ $\Rightarrow$ $r =$ \underline{\hspace{2cm}} cm

        \vspace{0.5cm}
        \item Ein Zylinder hat die Oberfläche $O = 54\pi$ cm$^2$ und den Radius $r = 3$ cm. Berechne die Höhe.

        $O = 2\pi r^2 + 2\pi rh = 2\pi r(r + h)$

        $54\pi = 2\pi \cdot 3(3 + h) = 6\pi(3 + h)$

        $9 = 3 + h$ $\Rightarrow$ $h =$ \underline{\hspace{2cm}} cm
    \end{enumerate}

    \vspace{1cm}

    \item \textbf{Sachaufgaben:}
    \vspace{0.5cm}

    \begin{enumerate}[label=\alph*)]
        \item \textbf{Konservendose:} Eine zylindrische Konservendose hat einen Durchmesser von 8 cm und eine Höhe von 12 cm. Wie viel Blech wird für die Dose benötigt (Oberfläche)?

        $r = 4$ cm, $h = 12$ cm

        $O = 2\pi r^2 + 2\pi rh = 2\pi \cdot 16 + 2\pi \cdot 4 \cdot 12 = 32\pi + 96\pi = 128\pi \approx$ \underline{\hspace{2cm}} cm$^2$

        \vspace{0.5cm}
        \item \textbf{Wassertank:} Ein zylindrischer Wassertank hat einen Radius von 1{,}5 m und eine Höhe von 3 m. Wie viel Liter Wasser passen hinein?

        $V = \pi r^2 h = \pi \cdot 1{,}5^2 \cdot 3 = \pi \cdot 2{,}25 \cdot 3 = 6{,}75\pi \approx$ \underline{\hspace{2cm}} m$^3$

        In Litern: $V \approx 21{,}21$ m$^3$ = \underline{\hspace{2cm}} Liter (1 m$^3$ = 1000 Liter)

        \vspace{0.5cm}
        \item \textbf{Röhre:} Eine Röhre ist 50 cm lang und hat einen Innendurchmesser von 6 cm. Welches Volumen kann durch die Röhre fließen?

        $r = 3$ cm, $h = 50$ cm

        $V = \pi \cdot 3^2 \cdot 50 = 450\pi \approx$ \underline{\hspace{2cm}} cm$^3$
    \end{enumerate}

    \vspace{1cm}

    \item \textbf{Komplexe Aufgabe:}
    \vspace{0.5cm}

    Ein zylindrisches Rohr hat einen Außenradius von $R = 5$ cm, einen Innenradius von $r = 3$ cm und eine Länge von $h = 20$ cm.

    \begin{enumerate}[label=\alph*)]
        \item Volumen des Rohrmaterials:
        $V = \pi R^2 h - \pi r^2 h = \pi h(R^2 - r^2) = \pi \cdot 20 \cdot (25 - 9) = 320\pi \approx$ \underline{\hspace{2cm}} cm$^3$

        \vspace{0.3cm}
        \item Oberfläche des Rohrs (außen + innen + 2 Ringe):
        $O = 2\pi R h + 2\pi r h + 2\pi(R^2 - r^2) =$ \underline{\hspace{5cm}} cm$^2$
    \end{enumerate}

\end{enumerate}