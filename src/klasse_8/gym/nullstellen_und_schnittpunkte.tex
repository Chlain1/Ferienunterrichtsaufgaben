\subsection*{Teil C: Nullstellen und Schnittpunkte (30 Minuten)}

\begin{enumerate}[label=\arabic*., resume]

    \item \textbf{Bestimme die Nullstelle (setze $y = 0$):}
    \vspace{0.5cm}

    \textit{Beispiel:} $y = 2x - 4$, setze $0 = 2x - 4$, dann $2x = 4$, also $x = 2$

    \begin{enumerate}[label=\alph*)]
        \item $y = 3x - 6$: $0 = 3x - 6$, also $3x =$ \underline{\hspace{1cm}}, $x =$ \underline{\hspace{1cm}}
        \vspace{0.5cm}
        \item $y = -2x + 8$: \underline{\hspace{6cm}}
        \vspace{0.5cm}
        \item $y = 4x + 12$: \underline{\hspace{6cm}}
    \end{enumerate}

    \vspace{1cm}

    \item \textbf{Sachaufgabe:}
    \vspace{0.5cm}

    Ein Handy-Tarif kostet 10€ Grundgebühr plus 0,20€ pro Minute.

    \begin{enumerate}[label=\alph*)]
        \item Stelle eine Funktionsgleichung für die Kosten $K$ in Abhängigkeit der Minuten $m$ auf:
        \vspace{0.3cm}

        $K(m) =$ \underline{\hspace{4cm}}

        \item Wie viel kostet der Tarif bei 50 Minuten Telefonzeit?
        \vspace{0.3cm}

        $K(50) =$ \underline{\hspace{4cm}} = \underline{\hspace{2cm}}€

        \item Bei wie vielen Minuten betragen die Kosten genau 20€?
        \vspace{0.3cm}

        $20 = 10 + 0,20m$, also $m =$ \underline{\hspace{2cm}} Minuten
    \end{enumerate}

    \vspace{1cm}

    \item \textbf{Zusatzaufgabe:}

    Bestimme die Funktionsgleichung einer Geraden, die durch die Punkte $A(0|3)$ und $B(2|7)$ geht.

    \textit{Tipp: $t = 3$ (y-Achsenabschnitt), $m = \dfrac{7-3}{2-0} = \dfrac{4}{2} = 2$}

    $y =$ \underline{\hspace{3cm}}

\end{enumerate}