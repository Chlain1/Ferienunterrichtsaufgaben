\subsection*{Teil A: Terme, Funktionen und Gleichungen (30 Minuten)}

\begin{enumerate}[label=\arabic*.]

    \item \textbf{Termumformungen und Funktionen:}
    \vspace{0.5cm}

    \begin{enumerate}[label=\alph*)]
        \item Vereinfache: $3(2x - 1) + 2(x + 4) =$ \underline{\hspace{4cm}}
        \vspace{0.3cm}
        \item Bestimme die Nullstelle von $y = 3x - 9$: $x =$ \underline{\hspace{2cm}}
        \vspace{0.3cm}
        \item Die Gerade $y = 2x + b$ geht durch den Punkt $P(1|5)$. Bestimme $b$: $b =$ \underline{\hspace{2cm}}
    \end{enumerate}

    \vspace{1cm}

    \item \textbf{Bruchterme:}
    \vspace{0.5cm}

    \begin{enumerate}[label=\alph*)]
        \item Definitionsmenge von $\dfrac{3x}{x^2-9}$: $\mathbb{D} = \mathbb{R} \setminus \{$ \underline{\hspace{2cm}} $\}$
        \vspace{0.3cm}
        \item Berechne: $\dfrac{2}{x} + \dfrac{3}{x+1} =$ \underline{\hspace{4cm}}
        \vspace{0.3cm}
        \item Löse: $\dfrac{4}{x} = \dfrac{2}{3}$ $\Rightarrow$ $x =$ \underline{\hspace{2cm}}
    \end{enumerate}

    \vspace{1cm}

    \item \textbf{Gleichungssystem:}
    \vspace{0.5cm}

    Löse: $\begin{cases} 2x + y = 7 \\ x - y = 2 \end{cases}$

    \textit{Tipp: Additionsverfahren verwenden}

    Addition: $3x = 9$ $\Rightarrow$ $x =$ \underline{\hspace{1.5cm}}, $y =$ \underline{\hspace{1.5cm}}

\end{enumerate}