\subsection*{Teil A: Kreisberechnungen (30 Minuten)}

\begin{enumerate}[label=\arabic*.]

    \item \textbf{Berechne Umfang und Flächeninhalt der Kreise:}
    \vspace{0.5cm}

    \begin{tabular}{|c|c|c|c|}
        \hline
        Radius $r$ & Durchmesser $d$ & Umfang $U$ & Flächeninhalt $A$ \\
        \hline
        $3$ cm & $6$ cm & $U = 2\pi \cdot 3 = 6\pi \approx 18{,}84$ cm & $A = \pi \cdot 3^2 = 9\pi \approx$ \underline{\hspace{1.5cm}} cm$^2$ \\
        \hline
        $5$ cm & \underline{\hspace{1cm}} cm & \underline{\hspace{2cm}} cm & \underline{\hspace{2cm}} cm$^2$ \\
        \hline
        \underline{\hspace{1cm}} cm & $12$ cm & \underline{\hspace{2cm}} cm & \underline{\hspace{2cm}} cm$^2$ \\
        \hline
        $2{,}5$ cm & \underline{\hspace{1cm}} cm & \underline{\hspace{2cm}} cm & \underline{\hspace{2cm}} cm$^2$ \\
        \hline
    \end{tabular}

    \vspace{1cm}

    \item \textbf{Bestimme den fehlenden Wert:}
    \vspace{0.5cm}

    \textit{Beispiel:} Gegeben $U = 12\pi$ cm, gesucht $r$: $2\pi r = 12\pi$, also $r = 6$ cm

    \begin{enumerate}[label=\alph*)]
        \item Gegeben: $U = 8\pi$ cm, gesucht: $r$ = \underline{\hspace{2cm}} cm
        \vspace{0.3cm}
        \item Gegeben: $A = 25\pi$ cm$^2$, gesucht: $r$ = \underline{\hspace{2cm}} cm
        \vspace{0.3cm}
        \item Gegeben: $d = 14$ cm, gesucht: $A$ = \underline{\hspace{2cm}} cm$^2$
        \vspace{0.3cm}
        \item Gegeben: $A = 49\pi$ cm$^2$, gesucht: $U$ = \underline{\hspace{2cm}} cm
    \end{enumerate}

    \vspace{1cm}

    \item \textbf{Kreisringe und Kreissektoren:}
    \vspace{0.5cm}

    \begin{enumerate}[label=\alph*)]
        \item Ein Kreisring hat den Außenradius $R = 8$ cm und den Innenradius $r = 5$ cm.
        \vspace{0.3cm}

        Fläche des Kreisrings: $A = \pi R^2 - \pi r^2 = \pi(R^2 - r^2) = \pi(64 - 25) = 39\pi \approx$ \underline{\hspace{2cm}} cm$^2$

        \vspace{0.5cm}
        \item Ein Viertelkreis (90°-Sektor) hat den Radius $r = 6$ cm.
        \vspace{0.3cm}

        Bogenlänge: $b = \dfrac{90°}{360°} \cdot 2\pi r = \dfrac{1}{4} \cdot 2\pi \cdot 6 =$ \underline{\hspace{2cm}} cm

        Sektorfläche: $A = \dfrac{90°}{360°} \cdot \pi r^2 = \dfrac{1}{4} \cdot \pi \cdot 36 =$ \underline{\hspace{2cm}} cm$^2$
    \end{enumerate}

\end{enumerate}