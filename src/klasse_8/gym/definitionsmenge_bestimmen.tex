\subsection*{Teil A: Definitionsmenge bestimmen (30 Minuten)}

\begin{enumerate}[label=\arabic*.]

    \item \textbf{Bestimme die Definitionsmenge (Nenner darf nicht 0 werden):}
    \vspace{0.5cm}

    \textit{Beispiel:} $\dfrac{3}{x-2}$: $x - 2 \neq 0$, also $x \neq 2$, $\mathbb{D} = \mathbb{R} \setminus \{2\}$

    \begin{enumerate}[label=\alph*)]
        \item $\dfrac{5}{x-1}$: $x - 1 \neq 0$, also $x \neq$ \underline{\hspace{1cm}}, $\mathbb{D} = \mathbb{R} \setminus \{$ \underline{\hspace{1cm}} $\}$
        \vspace{0.5cm}
        \item $\dfrac{2x}{x+3}$: \underline{\hspace{6cm}}, $\mathbb{D} = \mathbb{R} \setminus \{$ \underline{\hspace{1cm}} $\}$
        \vspace{0.5cm}
        \item $\dfrac{x-1}{2x-6}$: $2x - 6 \neq 0$, also $x \neq$ \underline{\hspace{1cm}}, $\mathbb{D} = \mathbb{R} \setminus \{$ \underline{\hspace{1cm}} $\}$
        \vspace{0.5cm}
        \item $\dfrac{4}{x^2-4}$: $x^2 - 4 \neq 0$, also $x \neq \pm$ \underline{\hspace{1cm}}, $\mathbb{D} = \mathbb{R} \setminus \{$ \underline{\hspace{2cm}} $\}$
    \end{enumerate}

    \vspace{1cm}

    \item \textbf{Erweitern und Kürzen von Bruchtermen:}
    \vspace{0.5cm}

    \begin{tabular}{ll}
        a) $\dfrac{2x}{3} \cdot \dfrac{2}{2}$ = $\dfrac{4x}{6}$ & b) $\dfrac{6x^2}{9x}$ = $\dfrac{6x^2 : 3x}{9x : 3x}$ = \underline{\hspace{2cm}} \\[4ex]
        c) $\dfrac{4x}{x^2}$ = \underline{\hspace{2cm}} & d) $\dfrac{15xy}{25x}$ = \underline{\hspace{2cm}}
    \end{tabular}

    \vspace{1cm}

    \item \textbf{Bestimme den Hauptnenner:}
    \vspace{0.5cm}

    \textit{Beispiel:} $\dfrac{1}{x-1}$ und $\dfrac{2}{x+1}$: Hauptnenner = $(x-1)(x+1)$

    \begin{enumerate}[label=\alph*)]
        \item $\dfrac{3}{x}$ und $\dfrac{2}{x+2}$: Hauptnenner = \underline{\hspace{3cm}}
        \vspace{0.5cm}
        \item $\dfrac{1}{2x}$ und $\dfrac{3}{4x}$: Hauptnenner = \underline{\hspace{3cm}}
        \vspace{0.5cm}
        \item $\dfrac{2}{x-3}$ und $\dfrac{1}{x+1}$: Hauptnenner = \underline{\hspace{3cm}}
    \end{enumerate}

\end{enumerate}