\subsection*{Teil B: Geometrie - Kreis und Anwendungen (30 Minuten)}

\begin{enumerate}[label=\arabic*., resume]

    \item \textbf{Kreisberechnungen:}
    \vspace{0.5cm}

    Ein Kreis hat den Radius $r = 6$ cm.

        \begin{enumerate}[label=\alph*)]
        \item Umfang: $U = 2\pi r =$ \underline{\hspace{2cm}} cm
        \vspace{0.3cm}
        \item Flächeninhalt: $A = \pi r^2 =$ \underline{\hspace{2cm}} cm$^2$
        \vspace{0.3cm}
        \item Ein Halbkreis mit diesem Radius hat die Fläche: $A = \dfrac{1}{2}\pi r^2 =$ \underline{\hspace{2cm}} cm$^2$
    \end{enumerate}

    \vspace{1cm}

    \item \textbf{Zylinder:}
    \vspace{0.5cm}

    Ein Zylinder hat $r = 4$ cm und $h = 9$ cm.

    \begin{enumerate}[label=\alph*)]
        \item Grundfläche: $G =$ \underline{\hspace{2cm}} cm$^2$
        \vspace{0.3cm}
        \item Mantelfläche: $M = 2\pi rh =$ \underline{\hspace{2cm}} cm$^2$
        \vspace{0.3cm}
        \item Volumen: $V = \pi r^2 h =$ \underline{\hspace{2cm}} cm$^3$
    \end{enumerate}

    \vspace{1cm}

    \item \textbf{Umgekehrte Berechnung:}
    \vspace{0.5cm}

    Ein Zylinder hat das Volumen $V = 72\pi$ cm$^3$ und den Radius $r = 3$ cm.

    Bestimme die Höhe: $V = \pi r^2 h$ $\Rightarrow$ $72\pi = \pi \cdot 9 \cdot h$ $\Rightarrow$ $h =$ \underline{\hspace{2cm}} cm

\end{enumerate}
