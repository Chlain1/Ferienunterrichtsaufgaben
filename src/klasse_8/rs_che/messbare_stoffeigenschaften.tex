\subsection*{Teil B: Messbare Stoffeigenschaften (30 Minuten)}

\begin{enumerate}[label=\arabic*.]
    \setcounter{enumi}{3}

    \item \textbf{Ergänze die fehlenden Begriffe:}
    \vspace{0.5cm}

    \begin{tabular}{|p{6cm}|p{8cm}|}
        \hline
        \textbf{Eigenschaft} & \textbf{Wie wird sie gemessen?} \\
        \hline
        Schmelztemperatur & Mit einem \underline{\hspace{3cm}} \\
        \hline
        Dichte & Masse durch \underline{\hspace{3cm}} \\
        \hline
        Elektrische Leitfähigkeit & Mit einem \underline{\hspace{3cm}} \\
        \hline
        Löslichkeit & Durch \underline{\hspace{3cm}} in Wasser \\
        \hline
    \end{tabular}

    \vspace{1cm}

    \item \textbf{Ordne die Stoffe nach ihrer Schmelztemperatur (von niedrig zu hoch):}
    \vspace{0.5cm}

    Eis (0°C), Eisen (1538°C), Wachs (60°C), Kochsalz (801°C)

    1. \underline{\hspace{3cm}} \hspace{1cm}
    2. \underline{\hspace{3cm}} \hspace{1cm}
    3. \underline{\hspace{3cm}} \hspace{1cm}
    4. \underline{\hspace{3cm}}

    \vspace{1cm}

    \item \textbf{Kreuze an: Welche Stoffe leiten Strom?}
    \vspace{0.5cm}

    \begin{tabular}{|p{3cm}|p{2cm}|p{2cm}|}
        \hline
        \textbf{Stoff} & \textbf{Ja} & \textbf{Nein} \\
        \hline
        Kupfer & & \\
        \hline
        Holz & & \\
        \hline
        Salzwasser & & \\
        \hline
        Gummi & & \\
        \hline
        Aluminium & & \\
        \hline
    \end{tabular}

\end{enumerate}
