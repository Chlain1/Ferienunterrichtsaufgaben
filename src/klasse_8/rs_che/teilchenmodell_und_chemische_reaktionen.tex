\subsection*{Teil B: Teilchenmodell und chemische Reaktionen (30 Minuten)}

\begin{enumerate}[label=\arabic*.]
    \setcounter{enumi}{2}

    \item \textbf{Erkläre mit dem Teilchenmodell und stelle Wortgleichungen auf:}
    \vspace{0.5cm}

    \begin{enumerate}[label=\alph*)]
        \item \textbf{Eisenwolle verbrennt:}

        \textbf{Beobachtung:} Eisenwolle glüht, wird schwerer, schwarzes Pulver entsteht

        \textbf{Teilchenebene:} \underline{\hspace{8cm}}
        \vspace{1cm}

        \textbf{Wortgleichung:} \underline{\hspace{3cm}} + \underline{\hspace{3cm}} $\rightarrow$ \underline{\hspace{3cm}}
        \vspace{1cm}

        \item \textbf{Wasser wird elektrolysiert:}

        \textbf{Beobachtung:} Gasbildung an beiden Elektroden, verschiedene Gase

        \textbf{Teilchenebene:} \underline{\hspace{8cm}}
        \vspace{1cm}

        \textbf{Wortgleichung:} \underline{\hspace{3cm}} $\rightarrow$ \underline{\hspace{3cm}} + \underline{\hspace{3cm}}
        \vspace{1cm}

        \item \textbf{Magnesium verbrennt:}

        \textbf{Beobachtung:} Helles weißes Licht, weißes Pulver entsteht

        \textbf{Teilchenebene:} \underline{\hspace{8cm}}
        \vspace{1cm}

        \textbf{Wortgleichung:} \underline{\hspace{3cm}} + \underline{\hspace{3cm}} $\rightarrow$ \underline{\hspace{3cm}}
        \vspace{1cm}
    \end{enumerate}

    \vspace{1cm}

    \item \textbf{Aggregatzustände und Teilchenbewegung:}
    \vspace{0.5cm}

    \begin{enumerate}[label=\alph*)]
        \item Bei welcher Temperatur bewegen sich Teilchen am langsamsten? \underline{\hspace{4cm}}

        \item In welchem Aggregatzustand sind die Anziehungskräfte zwischen Teilchen am stärksten? \underline{\hspace{4cm}}

        \item Warum kann man Flüssigkeiten nicht zusammendrücken? 
        \vspace{1cm}

        \item Erkläre, warum Parfüm sich im Raum verteilt:
        \vspace{1cm}
    \end{enumerate}

\end{enumerate}