\subsection*{Teil B: Nachweisreaktionen (30 Minuten)}

\begin{enumerate}[label=\arabic*.]
    \setcounter{enumi}{3}

    \item \textbf{Vervollständige die Nachweistabelle:}
    \vspace{0.5cm}

    \begin{tabular}{|p{3cm}|p{4cm}|p{4cm}|}
        \hline
        \textbf{Gas/Stoff} & \textbf{Nachweisverfahren} & \textbf{Beobachtung} \\
        \hline
        Sauerstoff & \underline{\hspace{3cm}} & Holzspan \underline{\hspace{2cm}} auf \\
        \hline
        Wasserstoff & \underline{\hspace{3cm}} & Knall, \underline{\hspace{2cm}} Flamme \\
        \hline
        Kohlenstoffdioxid & \underline{\hspace{3cm}} & Kalkwasser wird \underline{\hspace{2cm}} \\
        \hline
        Wasser & \underline{\hspace{3cm}} & Cobaltchlorid wird von \underline{\hspace{1cm}} zu \underline{\hspace{1cm}} \\
        \hline
    \end{tabular}

    \vspace{1cm}

    \item \textbf{Ordne die Sätze den richtigen Gasen zu:}
    \vspace{0.5cm}

    \textbf{Gase:} Sauerstoff, Wasserstoff, Kohlenstoffdioxid

    \begin{enumerate}[label=\alph*)]
        \item Dieses Gas entsteht bei der Verbrennung: \underline{\hspace{3cm}}
        \item Dieses Gas ist nötig für die Verbrennung: \underline{\hspace{3cm}}
        \item Dieses Gas ist das leichteste aller Gase: \underline{\hspace{3cm}}
        \item Dieses Gas wird bei der Fotosynthese verbraucht: \underline{\hspace{3cm}}
        \item Dieses Gas wird bei der Elektrolyse von Wasser produziert: \underline{\hspace{3cm}}
    \end{enumerate}

    \vspace{1cm}

    \item \textbf{Beschreibe das Experiment:}
    \vspace{0.5cm}

    Du willst nachweisen, ob in einem Getränk Kohlenstoffdioxid enthalten ist.

    \textbf{Material:} \underline{\hspace{8cm}}

    \textbf{Durchführung:} 
    \vspace{2cm}

    \textbf{Beobachtung bei positivem Nachweis:}
    \vspace{1cm}

\end{enumerate}