\subsection*{Teil A: Chemische Reaktion erkennen (30 Minuten)}

\begin{enumerate}[label=\arabic*.]
    \item \textbf{Kreuze an: Ist es eine chemische Reaktion?}
    \vspace{0.5cm}

    \begin{tabular}{|p{7cm}|p{2cm}|p{2cm}|}
        \hline
        \textbf{Vorgang} & \textbf{Ja} & \textbf{Nein} \\
        \hline
        Wasser gefriert zu Eis & & \\
        \hline
        Holz verbrennt zu Asche & & \\
        \hline
        Zucker löst sich in Wasser & & \\
        \hline
        Eisen rostet & & \\
        \hline
        Wasser verdunstet & & \\
        \hline
        Magnesium verbrennt mit weißem Licht & & \\
        \hline
        Salz wird zerkleinert & & \\
        \hline
    \end{tabular}

    \vspace{1cm}

    \item \textbf{Was sind die Kennzeichen einer chemischen Reaktion?}
    \vspace{0.5cm}

    Setze ein: \textbf{Farbänderung, Gasbildung, neue Stoffe, Energieaufnahme, Wärmeabgabe}

    \begin{enumerate}[label=\alph*)]
        \item Bei chemischen Reaktionen entstehen \underline{\hspace{4cm}}.
        \item Oft ist eine \underline{\hspace{4cm}} zu beobachten.
        \item Es kann zur \underline{\hspace{4cm}} kommen.
        \item Meist erfolgt \underline{\hspace{4cm}} oder \underline{\hspace{4cm}}.
    \end{enumerate}

    \vspace{1cm}

    \item \textbf{Benenne Ausgangsstoffe und Produkte:}
    \vspace{0.5cm}

    \textbf{Beispiel:} Wasserstoff + Sauerstoff $\rightarrow$ Wasser

    \begin{enumerate}[label=\alph*)]
        \item Ausgangsstoffe (Edukte): \underline{\hspace{5cm}}
        \item Produkte: \underline{\hspace{5cm}}
    \end{enumerate}

    \vspace{0.5cm}

    \textbf{Aufgabe:} Kohlenstoff + Sauerstoff $\rightarrow$ Kohlenstoffdioxid

    \begin{enumerate}[label=\alph*)]
        \item Ausgangsstoffe (Edukte): \underline{\hspace{5cm}}
        \item Produkte: \underline{\hspace{5cm}}
    \end{enumerate}

\end{enumerate}