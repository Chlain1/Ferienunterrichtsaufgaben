\subsection*{Teil B: Aggregatzustandsänderungen (30 Minuten)}

\begin{enumerate}[label=\arabic*.]
    \setcounter{enumi}{3}

    \item \textbf{Benenne die Übergänge:}
    \vspace{0.5cm}

    \begin{center}
        \begin{tabular}{ccc}
            \textbf{fest} & $\xrightarrow{\text{\underline{\hspace{2cm}}}}$ & \textbf{flüssig} \\[1cm]
            & & \\
            $\downarrow$ \underline{\hspace{2cm}} & & $\uparrow$ \underline{\hspace{2cm}} \\[1cm]
            & & \\
            \textbf{gasförmig} & $\xleftarrow{\text{\underline{\hspace{2cm}}}}$ & 
        \end{tabular}
    \end{center}

    \vspace{1cm}

    \item \textbf{Erkläre mit dem Teilchenmodell:}
    \vspace{0.5cm}

    \begin{enumerate}[label=\alph*)]
        \item Warum dehnt sich Luft beim Erwärmen aus?
        \vspace{2cm}

        \item Warum kann man Gase komprimieren, Feststoffe aber nicht?
        \vspace{2cm}

        \item Warum verdunstet Wasser auch bei Raumtemperatur?
        \vspace{2cm}
    \end{enumerate}

    \vspace{1cm}

    \item \textbf{Ordne die Temperaturen zu:}
    \vspace{0.5cm}

    \textbf{Temperaturen:} 0°C, 100°C, -273°C

    \begin{enumerate}[label=\alph*)]
        \item Schmelztemperatur von Eis: \underline{\hspace{3cm}}
        \item Siedetemperatur von Wasser: \underline{\hspace{3cm}}
        \item Absoluter Nullpunkt (Teilchen stehen still): \underline{\hspace{3cm}}
    \end{enumerate}

\end{enumerate}