\subsection*{Teil C: Teilchenmodell im Alltag (30 Minuten)}

\begin{enumerate}[label=\arabic*.]
    \setcounter{enumi}{6}

    \item \textbf{Erkläre folgende Beobachtungen mit dem Teilchenmodell:}
    \vspace{0.5cm}

    \begin{enumerate}[label=\alph*)]
        \item Ein Luftballon platzt, wenn er zu stark aufgepumpt wird.
        \vspace{2cm}

        \item Parfüm verteilt sich im ganzen Raum.
        \vspace{2cm}

        \item Wasser gefriert zu Eis und dehnt dabei das Gefäß aus.
        \vspace{2cm}

        \item Ein Metallstab wird länger, wenn er erhitzt wird.
        \vspace{2cm}
    \end{enumerate}

    \vspace{1cm}

    \item \textbf{Kreuze die richtigen Aussagen an:}
    \vspace{0.5cm}

    \begin{tabularx}{\textwidth}{|l|c|}
        \hline
        \textbf{Aussage} & \textbf{Richtig} \\
        \hline
        In Gasen bewegen sich die Teilchen am schnellsten. & \\
        \hline
        Beim Schmelzen werden die Teilchen größer. & \\
        \hline
        In Feststoffen sind die Teilchen starr angeordnet. & \\
        \hline
        Teilchen können sich durch andere Teilchen hindurchbewegen. & \\
        \hline
        Bei höherer Temperatur bewegen sich Teilchen langsamer. & \\
        \hline
        Zwischen den Teilchen ist leerer Raum. & \\
        \hline
    \end{tabularx}

\end{enumerate}