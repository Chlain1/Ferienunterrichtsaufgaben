\subsection*{Teil C: Laborgeräte und Sicherheit (30 Minuten)}

\begin{enumerate}[label=\arabic*.]
    \setcounter{enumi}{6}

    \item \textbf{Benenne die Laborgeräte:}
    \vspace{0.5cm}

    \begin{tabular}{|p{4cm}|p{10cm}|}
        \hline
        \textbf{Gerät} & \textbf{Verwendung} \\
        \hline
        \underline{\hspace{3cm}} & Zum Erhitzen von Flüssigkeiten \\
        \hline
        \underline{\hspace{3cm}} & Zum Abmessen genauer Volumina \\
        \hline
        \underline{\hspace{3cm}} & Zum Filtrieren \\
        \hline
        \underline{\hspace{3cm}} & Zum Erhitzen fester Stoffe \\
        \hline
        \underline{\hspace{3cm}} & Zum Destillieren \\
        \hline
    \end{tabular}

    \vspace{1cm}

    \item \textbf{Gefahrensymbole - was bedeuten sie?}
    \vspace{0.5cm}

    Ordne zu: \textbf{ätzend, brandfördernd, explosiv, giftig, umweltgefährlich}

    \begin{enumerate}[label=\alph*)]
        \item GHS01 - Bombe: \underline{\hspace{4cm}}
        \item GHS03 - Flamme über Kreis: \underline{\hspace{4cm}}
        \item GHS05 - Ätzwirkung: \underline{\hspace{4cm}}
        \item GHS06 - Totenkopf: \underline{\hspace{4cm}}
        \item GHS09 - Baum und Fisch: \underline{\hspace{4cm}}
    \end{enumerate}

    \vspace{1cm}

    \item \textbf{Sicherheitsregeln - richtig oder falsch?}
    \vspace{0.5cm}

    \begin{tabularx}{\textwidth}{|X|c|c|}
        \hline
        \textbf{Regel} & \textbf{Richtig} & \textbf{Falsch} \\
        \hline
        Schutzbrille nur bei gefährlichen Experimenten tragen. & & \\
        \hline
        Chemikalien niemals mit dem Mund pipettieren. & & \\
        \hline
        Hände nach dem Experimentieren waschen. & & \\
        \hline
        Reste können in den Ausguss geschüttet werden. & & \\
        \hline
        Bei Hautkontakt mit Säure sofort mit Wasser spülen. & & \\
        \hline
    \end{tabularx}

\end{enumerate}