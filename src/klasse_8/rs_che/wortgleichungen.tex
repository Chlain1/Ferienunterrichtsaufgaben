\subsection*{Teil B: Wortgleichungen aufstellen (30 Minuten)}

\begin{enumerate}[label=\arabic*.]
    \setcounter{enumi}{3}

    \item \textbf{Vervollständige die Wortgleichungen:}
    \vspace{0.5cm}

    \begin{enumerate}[label=\alph*)]
        \item Magnesium + \underline{\hspace{4cm}} $\rightarrow$ Magnesiumoxid
        \vspace{0.5cm}

        \item \underline{\hspace{4cm}} + Sauerstoff $\rightarrow$ Kupferoxid
        \vspace{0.5cm}

        \item Wasserstoff + \underline{\hspace{4cm}} $\rightarrow$ Wasser
        \vspace{0.5cm}

        \item \underline{\hspace{4cm}} + Sauerstoff $\rightarrow$ Eisenoxid
        \vspace{0.5cm}

        \item Schwefel + \underline{\hspace{4cm}} $\rightarrow$ Schwefeldioxid
        \vspace{0.5cm}
    \end{enumerate}

    \vspace{1cm}

    \item \textbf{Stelle selbst Wortgleichungen auf:}
    \vspace{0.5cm}

    \begin{enumerate}[label=\alph*)]
        \item Aluminium reagiert mit Sauerstoff zu Aluminiumoxid.
        \vspace{1cm}
        \underline{\hspace{4cm}} + \underline{\hspace{4cm}} $\rightarrow$ \underline{\hspace{4cm}}

        \item Zink reagiert mit Sauerstoff zu Zinkoxid.
        \vspace{1cm}
        \underline{\hspace{4cm}} + \underline{\hspace{4cm}} $\rightarrow$ \underline{\hspace{4cm}}

        \item Calcium reagiert mit Sauerstoff zu Calciumoxid.
        \vspace{1cm}
        \underline{\hspace{4cm}} + \underline{\hspace{4cm}} $\rightarrow$ \underline{\hspace{4cm}}
    \end{enumerate}

    \vspace{1cm}

    \item \textbf{Erkenne das Reaktionsmuster:}
    \vspace{0.5cm}

    Alle obigen Reaktionen folgen dem Schema:

    \textbf{Metall} + \textbf{Sauerstoff} $\rightarrow$ \textbf{\underline{\hspace{4cm}}}

    Diese Art von Reaktion nennt man \underline{\hspace{4cm}}.

\end{enumerate}