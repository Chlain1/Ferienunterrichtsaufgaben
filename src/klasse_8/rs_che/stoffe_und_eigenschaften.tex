\subsection*{Teil A: Stoffe und Eigenschaften - Wiederholung (30 Minuten)}

\begin{enumerate}[label=\arabic*.]
    \item \textbf{Ergänze die Tabelle mit den wichtigsten Stoffen:}
    \vspace{0.5cm}

    \begin{tabular}{|p{3cm}|p{2cm}|p{2cm}|p{3cm}|p{3cm}|}
        \hline
        \textbf{Stoff} & \textbf{Farbe} & \textbf{Zustand} & \textbf{Löslich in H$_2$O} & \textbf{Besonderheit} \\
        \hline
        Sauerstoff & & & & Nötig für \underline{\hspace{2cm}} \\
        \hline
        Kohlenstoffdioxid & & & & Nachweis mit \underline{\hspace{2cm}} \\
        \hline
        Wasserstoff & & & & \underline{\hspace{2cm}} aller Gase \\
        \hline
        Eisen & & & & Wird von \underline{\hspace{2cm}} angezogen \\
        \hline
        Kupfer & & & & Leitet \underline{\hspace{2cm}} \\
        \hline
        Natriumchlorid & & & & Kochsalz, \underline{\hspace{2cm}} Geschmack \\
        \hline
    \end{tabular}

    \vspace{1cm}

    \item \textbf{Trenne das Gemisch Schritt für Schritt:}
    \vspace{0.5cm}

    \textbf{Gegeben:} Gemisch aus Sand, Eisenspänen, Salz und Wasser

    \textbf{Ziel:} Alle vier Stoffe getrennt gewinnen

    \begin{enumerate}[label=\arabic*.)]
        \item Schritt: \underline{\hspace{8cm}}
        \vspace{0.5cm}

        \textbf{Begründung:} \underline{\hspace{8cm}}
        \vspace{0.5cm}

        \item Schritt: \underline{\hspace{8cm}}
        \vspace{0.5cm}

        \textbf{Begründung:} \underline{\hspace{8cm}}
        \vspace{0.5cm}

        \item Schritt: \underline{\hspace{8cm}}
        \vspace{0.5cm}

        \textbf{Begründung:} \underline{\hspace{8cm}}
        \vspace{0.5cm}

        \item Schritt: \underline{\hspace{8cm}}
        \vspace{0.5cm}

        \textbf{Begründung:} \underline{\hspace{8cm}}
        \vspace{0.5cm}
    \end{enumerate}

\end{enumerate}