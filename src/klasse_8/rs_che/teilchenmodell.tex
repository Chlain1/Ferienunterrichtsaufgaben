\subsection*{Teil A: Das Teilchenmodell (30 Minuten)}

\begin{enumerate}[label=\arabic*.]
    \item \textbf{Vervollständige die Grundlagen des Teilchenmodells:}
    \vspace{0.5cm}

    \begin{enumerate}[label=\alph*)]
        \item Alle Stoffe bestehen aus kleinsten \underline{\hspace{4cm}}.
        \item Die Teilchen sind ständig in \underline{\hspace{4cm}}.
        \item Zwischen den Teilchen herrschen \underline{\hspace{4cm}}.
        \item Je höher die Temperatur, desto \underline{\hspace{4cm}} bewegen sich die Teilchen.
    \end{enumerate}

    \vspace{1cm}

    \item \textbf{Zeichne die Teilchenanordnung in den verschiedenen Aggregatzuständen:}
    \vspace{0.5cm}

    \begin{tabular}{|p{4cm}|p{4cm}|p{4cm}|}
        \hline
        \textbf{fest} & \textbf{flüssig} & \textbf{gasförmig} \\
        \hline
        & & \\[3cm]
        \hline
    \end{tabular}

    \vspace{1cm}

    \item \textbf{Ordne die Eigenschaften den Aggregatzuständen zu:}
    \vspace{0.5cm}

    \textbf{Eigenschaften:} feste Form, nimmt Gefäßform an, füllt ganzen Raum aus, nicht komprimierbar, leicht komprimierbar, schwer komprimierbar

    \begin{tabular}{|p{4cm}|p{10cm}|}
        \hline
        \textbf{fest} & \\[1cm]
        \hline
        \textbf{flüssig} & \\[1cm]
        \hline
        \textbf{gasförmig} & \\[1cm]
        \hline
    \end{tabular}

\end{enumerate}