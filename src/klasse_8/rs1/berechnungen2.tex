\subsection*{Teil D: Berechnungen und Anwendungen (20 Minuten)}

\begin{enumerate}[label=\arabic*.,resume]

    \item \textbf{Volumen und Oberfläche:}

    Berechne Volumen und Oberfläche eines Zylinders mit $r = 4$ cm und $h = 10$ cm.

    \textit{Formeln:} $V = \pi r^2 h$ \hspace{2cm} $O = 2\pi r^2 + 2\pi r h$

    \vspace{0.5cm}

    Volumen: $V = $ \underline{\hspace{8cm}}

    \vspace{1.5cm}

    Oberfläche: $O = $ \underline{\hspace{8cm}}

    \vspace{1.5cm}

    \item \textbf{Anwendung - Dose:}

    Eine zylindrische Konservendose hat einen Durchmesser von 8 cm und ist 12 cm hoch.

    \begin{enumerate}[label=\alph*)]
        \item Wie viel Inhalt fasst die Dose?

        \vspace{2cm}

        \item Wie viel Blech wird für die Dose benötigt (mit Deckel und Boden)?

        \vspace{2cm}

    \end{enumerate}

    \item \textbf{Rotationskörper-Anwendung:}

    Ein Kegel entsteht durch Rotation eines rechtwinkligen Dreiecks mit den Katheten 6 cm und 8 cm um die längere Kathete. Berechne das Volumen des Kegels.

    \textit{Formel:} $V = \frac{1}{3}\pi r^2 h$

    \vspace{3cm}

\end{enumerate}