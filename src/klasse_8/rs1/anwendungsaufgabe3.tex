\subsection*{Teil C: Anwendungsaufgaben (30 Minuten)}

\begin{enumerate}[label=\arabic*.,resume]

    \item \textbf{Handytarif:}

    Tarif A: 15 € Grundgebühr + 0,20 € pro Minute
    Tarif B: 8 € Grundgebühr + 0,35 € pro Minute

    \begin{enumerate}[label=\alph*)]
        \item Stelle die Funktionsgleichungen auf:

        Tarif A: $f_A(x) = $ \underline{\hspace{6cm}}

        Tarif B: $f_B(x) = $ \underline{\hspace{6cm}}

        \vspace{0.5cm}

        \item Berechne die Kosten für 60 Minuten bei beiden Tarifen:

        Tarif A: \underline{\hspace{6cm}}

        Tarif B: \underline{\hspace{6cm}}

        \vspace{0.5cm}

        \item Ab wie vielen Minuten ist Tarif A günstiger?

        \vspace{2cm}

    \end{enumerate}

    \item \textbf{Wassertank:}

    Ein Wassertank wird mit 50 Litern pro Minute gefüllt. Zu Beginn sind bereits 200 Liter im Tank.

    \begin{enumerate}[label=\alph*)]
        \item Funktionsgleichung: $V(t) = $ \underline{\hspace{6cm}}

        \item Wie viel Wasser ist nach 12 Minuten im Tank?

        \vspace{1.5cm}

        \item Nach wie vielen Minuten sind 800 Liter im Tank?

        \vspace{1.5cm}

    \end{enumerate}

\end{enumerate}