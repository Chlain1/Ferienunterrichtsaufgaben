\subsection*{Teil C: Berechnungen (25 Minuten)}

\begin{enumerate}[label=\arabic*.,resume]

    \item \textbf{Flächeninhalt berechnen:}

    \textit{Beispiel:} Parallelogramm mit $a = 8$ cm, $h_a = 5$ cm
    $A = a \cdot h_a = 8 \text{ cm} \cdot 5 \text{ cm} = 40 \text{ cm}^2$

    \vspace{0.5cm}

    \begin{enumerate}[label=\alph*)]
        \item Parallelogramm: $a = 12$ cm, $h_a = 7$ cm

        $A = $ \underline{\hspace{8cm}}

        \vspace{1cm}

        \item Trapez: $a = 10$ cm, $c = 6$ cm, $h = 4$ cm

        $A = \frac{(a+c) \cdot h}{2} = $ \underline{\hspace{8cm}}

        \vspace{1cm}

        \item Raute: $d_1 = 8$ cm, $d_2 = 6$ cm

        $A = \frac{d_1 \cdot d_2}{2} = $ \underline{\hspace{8cm}}

    \end{enumerate}

    \vspace{1cm}

    \item \textbf{Umfang berechnen:}

    \begin{enumerate}[label=\alph*)]
        \item Rechteck: $a = 15$ cm, $b = 8$ cm

        $U = $ \underline{\hspace{8cm}}

        \vspace{1cm}

        \item Quadrat: $a = 12$ cm

        $U = $ \underline{\hspace{8cm}}

    \end{enumerate}

\end{enumerate}