\subsection*{Teil C: Wahre Größe bestimmen (20 Minuten)}

\begin{enumerate}[label=\arabic*.,resume]

    \item \textbf{Aus Schrägbild die wahre Größe bestimmen:}

    Im Schrägbild eines Quaders (Verzerrungsmaßstab 1:2) misst du:
    - Vorderkante: 6 cm
    - Nach hinten verlaufende Kante: 2 cm  
    - Höhe: 4 cm

    \begin{enumerate}[label=\alph*)]
        \item Wahre Länge: \underline{\hspace{4cm}}

        \item Wahre Breite: \underline{\hspace{4cm}}

        \item Wahre Höhe: \underline{\hspace{4cm}}

        \item Berechne das wahre Volumen:

        \vspace{2cm}

    \end{enumerate}

    \item \textbf{Winkel und Strecken in Schrägbildern:}

    In einem Würfel mit der Kantenlänge $a = 5$ cm wird die Raumdiagonale eingezeichnet.

    \begin{enumerate}[label=\alph*)]
        \item Länge der Flächendiagonale: $d_F = a\sqrt{2} = $ \underline{\hspace{4cm}}

        \item Länge der Raumdiagonale: $d_R = a\sqrt{3} = $ \underline{\hspace{4cm}}

        \item Zeichne den Würfel mit eingezeichneter Raumdiagonale:

        \vspace{6cm}

    \end{enumerate}

\end{enumerate}