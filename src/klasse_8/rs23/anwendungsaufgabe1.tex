\subsection*{Teil D: Anwendungsaufgaben (10 Minuten)}

\begin{enumerate}[label=\arabic*.,resume]

    \item \textbf{Vermessung eines Grundstücks:}

    Ein dreieckiges Grundstück hat die Seitenlängen 40 m, 60 m und 80 m.

    \begin{enumerate}[label=\alph*)]
        \item Prüfe, ob diese Maße ein Dreieck ergeben können:

        \vspace{2cm}

        \item Der Landvermesser will das Grundstück kontrollieren. Er misst zwei Seiten (40 m und 60 m) und den eingeschlossenen Winkel (75°). Mit welchem Konstruktionsverfahren arbeitet er?

        \vspace{1.5cm}

        \item Berechne den Flächeninhalt des Dreiecks mit der Formel: $A = \frac{1}{2} \cdot a \cdot b \cdot \sin(\gamma)$

        $A = \frac{1}{2} \cdot 40 \cdot 60 \cdot \sin(75°) = $ \underline{\hspace{6cm}}

    \end{enumerate}

    \vspace{1cm}

    \item \textbf{Rechteckiges Schwimmbad:}

    Familie Weber plant ein rechteckiges Schwimmbad mit den Maßen 8 m × 5 m. Drumherum soll eine 1,5 m breite Terrasse gebaut werden.

    \begin{enumerate}[label=\alph*)]
        \item Welche Außenmaße hat die gesamte Anlage?

        \vspace{1.5cm}

        \item Berechne die Fläche der Terrasse:

        \vspace{2cm}

    \end{enumerate}

\end{enumerate}