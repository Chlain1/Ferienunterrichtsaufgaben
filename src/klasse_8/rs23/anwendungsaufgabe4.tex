\subsection*{Teil D: Anwendungsaufgaben (10 Minuten)}

\begin{enumerate}[label=\arabic*.,resume]

    \item \textbf{Schwimmbad:}

    Ein rechteckiges Schwimmbad ist 25 m lang, 12 m breit und durchschnittlich 2 m tief.

    \begin{enumerate}[label=\alph*)]
        \item Wie viele Liter Wasser passen hinein?

        \vspace{2cm}

        \item Die Beckenwände und der Boden sollen gefliest werden. Wie groß ist die zu fliesende Fläche?

        \vspace{2.5cm}

    \end{enumerate}

    \item \textbf{Verpackung:}

    Eine zylindrische Dose hat einen Durchmesser von 10 cm und eine Höhe von 15 cm.

    \begin{enumerate}[label=\alph*)]
        \item Inhalt der Dose in Litern:

        \vspace{2cm}

        \item Für die Verpackung wird die Dose in einen quaderförmigen Karton gestellt. Der Karton ist 12 cm × 12 cm × 16 cm groß. Wie viel Prozent des Kartons wird nicht genutzt?

        \vspace{3cm}

    \end{enumerate}

    \item \textbf{Materialberechnung:}

    Aus einem Holzbalken mit quadratischem Querschnitt (8 cm × 8 cm) und 3 m Länge soll ein zylindrischer Pfosten mit maximalem Durchmesser gedreht werden.

    \begin{enumerate}[label=\alph*)]
        \item Welchen Durchmesser hat der Zylinder?

        \vspace{2cm}

        \item Wie viel Holz geht dabei verloren? (in %)

        \vspace{2.5cm}

    \end{enumerate}

\end{enumerate}