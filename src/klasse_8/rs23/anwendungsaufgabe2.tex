\subsection*{Teil D: Anwendungsaufgaben (10 Minuten)}

\begin{enumerate}[label=\arabic*.,resume]

    \item \textbf{Zahlenrätsel:}

    Die Summe zweier aufeinanderfolgender gerader Zahlen ist 46. Bestimme die beiden Zahlen.

    \textit{Ansatz:} Erste Zahl: $x$, zweite Zahl: $x + 2$

    Gleichung: \underline{\hspace{8cm}}

    Lösung: Die Zahlen sind \underline{\hspace{3cm}} und \underline{\hspace{3cm}}

    \vspace{2cm}

    \item \textbf{Altersaufgabe:}

    Anna ist heute doppelt so alt wie ihr Bruder Ben. In 5 Jahren wird Anna 7 Jahre älter sein als Ben dann ist. Wie alt sind beide heute?

    \textit{Ansatz:} Ben heute: $x$ Jahre, Anna heute: $2x$ Jahre

    \vspace{3cm}

    Anna ist heute \underline{\hspace{3cm}} Jahre alt, Ben ist \underline{\hspace{3cm}} Jahre alt.

    \vspace{1cm}

    \item \textbf{Geometrie-Anwendung:}

    Die Seiten eines Rechtecks verhalten sich wie 3:5. Der Umfang beträgt 48 cm. Bestimme die Seitenlängen.

    \textit{Ansatz:} Seiten sind $3x$ und $5x$

    \vspace{3cm}

    Die Seiten sind \underline{\hspace{3cm}} cm und \underline{\hspace{3cm}} cm lang.

\end{enumerate}