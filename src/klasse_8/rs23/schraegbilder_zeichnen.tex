\subsection*{Teil A: Schrägbilder zeichnen (25 Minuten)}

\begin{enumerate}[label=\arabic*.]

    \item \textbf{Grundlagen des Schrägbilds:}

    Ergänze die Regeln für das Zeichnen von Schrägbildern:

    \begin{enumerate}[label=\alph*)]
        \item Die Vorderkante wird in \underline{\hspace{4cm}} gezeichnet.

        \item Der Verzerrungswinkel beträgt meist \underline{\hspace{2cm}}° oder \underline{\hspace{2cm}}°.

        \item Der Verzerrungsmaßstab beträgt meist \underline{\hspace{2cm}} oder \underline{\hspace{2cm}}.

        \item Nach hinten verlaufende Kanten werden \underline{\hspace{4cm}} dargestellt.
    \end{enumerate}

    \vspace{1cm}

    \item \textbf{Quader im Schrägbild:}

    Zeichne das Schrägbild eines Quaders mit den Maßen:
    Länge: 6 cm, Breite: 4 cm, Höhe: 3 cm
    (Verzerrungswinkel 45°, Verzerrungsmaßstab 1:2)

    \vspace{8cm}

    \item \textbf{Pyramide im Schrägbild:}

    Zeichne das Schrägbild einer quadratischen Pyramide:
    Grundkante: 4 cm, Höhe: 5 cm

    \vspace{7cm}

\end{enumerate}