% footer_template.tex - Vorlage für das Ende jedes Blattes
% Verwendung:
% \newcommand{\motivationstext}{Geschafft! Kontrolliere nochmal deine Lösungen!}
% \input{footer_template}

% Motivationsbox am Ende (wenn definiert)
\motivationsbox{Geschafft! Kontrolliere nochmal deine Lösungen!}

% Optionale Zusatzinformationen
\ifdefined\zusatzinfo
\vspace{0.5cm}
\begin{center}
\zusatzinfo
\end{center}
\fi

% Optionaler Platz für Notizen
\ifdefined\notizbereich
\vspace{1cm}
\noindent\textbf{Notizen / Fragen:}
\vspace{0.3cm}

\noindent\rule{\textwidth}{0.5pt}
\vspace{0.3cm}

\noindent\rule{\textwidth}{0.5pt}
\vspace{0.3cm}

\noindent\rule{\textwidth}{0.5pt}
\vspace{0.3cm}

\noindent\rule{\textwidth}{0.5pt}
\fi

% Optionale Selbsteinschätzung
\ifdefined\selbsteinschaetzung
\vspace{1cm}
\noindent\textbf{Selbsteinschätzung:} Wie gut konntest du die Aufgaben lösen?

\vspace{0.3cm}
\begin{center}
\begin{tabular}{|l|c|c|c|c|}
\hline
& sehr gut & gut & geht so & schwierig \\
\hline
Teil A & $\square$ & $\square$ & $\square$ & $\square$ \\
\hline
Teil B & $\square$ & $\square$ & $\square$ & $\square$ \\
\hline
Teil C & $\square$ & $\square$ & $\square$ & $\square$ \\
\hline
Teil D & $\square$ & $\square$ & $\square$ & $\square$ \\
\hline
\end{tabular}
\end{center}
\fi
