\subsection*{Teil C: Terme und Gleichungen (20 Minuten)}

\begin{enumerate}[resume, label=\arabic*.]
    \item \textbf{Vereinfache die Terme:}

    \begin{enumerate}[label=\alph*)]
        \item $2x + 3x = (2 + 3) \cdot x$ = \underline{\hspace{3cm}}
        \item $5a - 2a = (5 - 2) \cdot a$ = \underline{\hspace{3cm}}
        \item $4y + 2 + 3y - 1 = 7y + 1$ (ordne zuerst!) = \underline{\hspace{3cm}}
    \end{enumerate}

    \vspace{0.5cm}

    \item \textbf{Setze ein und berechne} ($x = 3$, $y = 2$):

    \begin{enumerate}[label=\alph*)]
        \item $2x + y = 2 \cdot 3 + 2$ = \underline{\hspace{3cm}}
        \item $3x - 2y = 3 \cdot 3 - 2 \cdot 2$ = \underline{\hspace{3cm}}
        \item $x^2 + y = 3^2 + 2 = 9 + 2$ = \underline{\hspace{3cm}}
    \end{enumerate}

    \vspace{0.5cm}

    \item \textbf{Sachaufgabe mit Term:}

    Ein Rechteck ist doppelt so lang wie breit. Die Breite ist $x$ cm.

    \begin{enumerate}[label=\alph*)]
        \item Stelle einen Term für die Länge auf: $l$ = \underline{\hspace{3cm}}
        \item Stelle einen Term für den Umfang auf: 
        $U = 2 \cdot (x + 2x) = 2 \cdot 3x$ = \underline{\hspace{3cm}}
        \item Berechne den Umfang für $x = 4$: 
        $U = 6 \cdot 4$ = \underline{\hspace{3cm}} cm
    \end{enumerate}
\end{enumerate}