\subsection*{Teil D: Textaufgaben mit Brüchen (15 Minuten)}

\begin{enumerate}[resume, label=\arabic*.]
    \item \textbf{Pizza-Aufgabe:} 

    Tom isst $\dfrac{2}{8}$ einer Pizza, Lisa isst $\dfrac{3}{8}$ der gleichen Pizza.

    \begin{enumerate}[label=\alph*)]
        \item Wie viel Pizza haben sie zusammen gegessen?

            \vspace{0.3cm}
            Rechnung: $\dfrac{2}{8} + \dfrac{3}{8}$ = \underline{\hspace{6cm}}

            \vspace{0.3cm}
            Antwort: \underline{\hspace{10cm}}

            \vspace{0.5cm}
        \item Wie viel Pizza ist noch übrig?

            \vspace{0.3cm}
            Rechnung: $1 - \dfrac{\phantom{00}}{\phantom{00}}$ = $\dfrac{8}{8} - \dfrac{\phantom{00}}{\phantom{00}}$ = \underline{\hspace{5cm}}

            \vspace{0.3cm}
            Antwort: \underline{\hspace{10cm}}
    \end{enumerate}

    \vspace{0.8cm}

    \item \textbf{Taschengeld:} 

    Max bekommt 12 € Taschengeld im Monat. Er gibt $\dfrac{1}{3}$ davon für Süßigkeiten aus.

    \vspace{0.5cm}
    \begin{itemize}
        \item[a)] Wie viel Euro gibt er für Süßigkeiten aus?

        \vspace{0.3cm}
        Rechnung: $\dfrac{1}{3}$ von 12 € = $12 : 3 \cdot 1$ = \underline{\hspace{5cm}}

        \vspace{0.3cm}
        Antwort: \underline{\hspace{10cm}}

        \vspace{0.5cm}
        \item[b)] Wie viel Euro hat er noch übrig?

        \vspace{0.3cm}
        Rechnung: 12 € $-$ \underline{\hspace{2cm}} € = \underline{\hspace{3cm}}

        \vspace{0.3cm}
        Antwort: \underline{\hspace{10cm}}
    \end{itemize}
\end{enumerate}