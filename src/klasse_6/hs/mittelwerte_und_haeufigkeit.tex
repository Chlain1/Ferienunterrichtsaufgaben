\subsection*{Teil B: Mittelwerte und Häufigkeiten (20 Minuten)}

\begin{enumerate}[resume, label=\arabic*.]
    \item \textbf{Berechne den Durchschnitt (arithmetisches Mittel):}
    \begin{enumerate}[label=\alph*)]
        \item Noten: 2, 3, 2, 4, 3, 2, 3

        Rechnung: $\dfrac{2 + 3 + 2 + 4 + 3 + 2 + 3}{7} = \dfrac{\phantom{19}}{7}$ = \underline{\hspace{3cm}}

        Durchschnittsnote: \underline{\hspace{3cm}}

        \vspace{0.5cm}
        \item Körpergröße in cm: 150, 155, 148, 162, 145

        Rechnung: $\dfrac{150 + 155 + 148 + 162 + 145}{5} = \dfrac{\phantom{760}}{5}$ = \underline{\hspace{3cm}}

        Durchschnittsgröße: \underline{\hspace{3cm}} cm
    \end{enumerate}

    \vspace{0.5cm}

    \item \textbf{Relative Häufigkeit:} 

    Von 20 Würfen mit einem Würfel kam die Zahl 4 genau 5 mal.

    Relative Häufigkeit für die 4: $\dfrac{5}{20} = \dfrac{\phantom{1}}{\phantom{4}}$ = 0,\underline{\hspace{1cm}} = \underline{\hspace{2cm}}\%
\end{enumerate}