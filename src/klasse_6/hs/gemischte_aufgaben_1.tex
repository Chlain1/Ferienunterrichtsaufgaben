\subsection*{Teil D: Gemischte Aufgaben (15 Minuten)}

\begin{enumerate}[resume, label=\arabic*.]
    \item \textbf{Rechne mit Dezimalzahlen:}

    \vspace{0.5cm}
    \begin{tabular}{ll}
        a) $3,5 + 2,7$ = \underline{\hspace{3cm}} & b) $8,2 - 3,6$ = \underline{\hspace{3cm}} \\[2ex]
        c) $4,5 \cdot 2$ = \underline{\hspace{3cm}} & d) $12,6 : 3$ = \underline{\hspace{3cm}}
    \end{tabular}

    \vspace{1cm}

    \item \textbf{Sachaufgabe:} 

    Ein Rezept für 4 Personen braucht $\dfrac{1}{2}$ Liter Milch.

    \begin{enumerate}[label=\alph*)]
        \item Wie viel Milch braucht man für 8 Personen?

        \vspace{0.3cm}
        \textit{Überlegung:} 8 Personen sind doppelt so viele wie 4 Personen.

        \vspace{0.3cm}
        Rechnung: $\dfrac{1}{2}$ Liter $\cdot 2$ = \underline{\hspace{5cm}}

        \vspace{0.3cm}
        Antwort: Man braucht \underline{\hspace{3cm}} Liter Milch.

        \vspace{0.5cm}
        \item Wie viel Milch braucht man für 2 Personen?

        \vspace{0.3cm}
        \textit{Überlegung:} 2 Personen sind halb so viele wie 4 Personen.

        \vspace{0.3cm}
        Rechnung: $\dfrac{1}{2}$ Liter $: 2$ = \underline{\hspace{5cm}}

        \vspace{0.3cm}
        Antwort: Man braucht \underline{\hspace{3cm}} Liter Milch.
    \end{enumerate}
\end{enumerate}