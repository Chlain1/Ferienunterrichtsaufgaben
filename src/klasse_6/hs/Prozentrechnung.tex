\subsection*{Teil D: Prozentrechnung (15 Minuten)}

\begin{enumerate}[resume, label=\arabic*.]
    \item \textbf{Wandle um:}

    \begin{tabular}{ll}
        a) 50\% = $\dfrac{50}{100} = \dfrac{1}{2}$ = 0,\underline{\hspace{1cm}} & 
        b) 25\% = $\dfrac{25}{100} = \dfrac{\phantom{00}}{\phantom{00}}$ = 0,\underline{\hspace{1cm}} \\[3ex]
        c) $\dfrac{1}{4}$ = \underline{\hspace{2cm}}\% & 
        d) 0,75 = $\dfrac{75}{100}$ = \underline{\hspace{2cm}}\%
    \end{tabular}

    \vspace{1cm}

    \item \textbf{Prozent berechnen:}

    \textit{Beispiel:} 10\% von 50 € = $\dfrac{10}{100} \cdot 50$ € = $\dfrac{1}{10} \cdot 50$ € = 5 €

    \begin{enumerate}[label=\alph*)]
        \item 50\% von 80 € = $\dfrac{1}{2} \cdot 80$ € = \underline{\hspace{3cm}}
        \item 10\% von 120 € = $\dfrac{1}{10} \cdot 120$ € = \underline{\hspace{3cm}}
        \item 25\% von 40 kg = $\dfrac{1}{4} \cdot 40$ kg = \underline{\hspace{3cm}}
    \end{enumerate}

    \vspace{0.5cm}

    \item \textbf{Rabatt-Aufgabe:}

    Ein Pullover kostet 40 €. Es gibt 20\% Rabatt.

    \begin{enumerate}[label=\alph*)]
        \item Wie viel € Rabatt bekommt man? 

        Rechnung: 20\% von 40 € = $\dfrac{20}{100} \cdot 40$ € = $\dfrac{1}{5} \cdot 40$ € = \underline{\hspace{3cm}}

        \item Was kostet der Pullover mit Rabatt? 

        Rechnung: 40 € $-$ \underline{\hspace{2cm}} € = \underline{\hspace{3cm}}
    \end{enumerate}
\end{enumerate}