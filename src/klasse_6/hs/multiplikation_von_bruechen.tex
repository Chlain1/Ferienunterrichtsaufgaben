\subsection*{Teil C: Multiplikation von Brüchen (25 Minuten)}

\begin{enumerate}[resume, label=\arabic*.]
    \item \textbf{Multipliziere die Brüche mit ganzen Zahlen:}

    \vspace{0.5cm}
    \begin{tabular}{l}
        a) $\dfrac{1}{2} \cdot 4 = \dfrac{1 \cdot 4}{2} = \dfrac{4}{2}$ = \underline{\hspace{3cm}} \\[3ex]
        b) $\dfrac{2}{3} \cdot 3 = \dfrac{2 \cdot 3}{3} = \dfrac{\phantom{00}}{\phantom{00}}$ = \underline{\hspace{3cm}} \\[3ex]
        c) $\dfrac{3}{4} \cdot 8 = \dfrac{\phantom{0000}}{\phantom{00}}$ = \underline{\hspace{3cm}}
    \end{tabular}

    \vspace{1cm}

    \item \textbf{Berechne den Bruchteil:}

    \textit{Beispiel:} $\dfrac{1}{4}$ von 12 = $12 : 4 \cdot 1 = 3$

    \vspace{0.5cm}
    \begin{tabular}{ll}
        a) $\dfrac{1}{2}$ von 20 = \underline{\hspace{3cm}} & 
        b) $\dfrac{1}{4}$ von 16 = \underline{\hspace{3cm}} \\[2ex]
        c) $\dfrac{2}{3}$ von 12 = \underline{\hspace{3cm}} & 
        d) $\dfrac{3}{5}$ von 10 = \underline{\hspace{3cm}}
    \end{tabular}

    \vspace{1cm}

    \item \textbf{Textaufgabe:} 

    In einer Klasse mit 24 Schülern sind $\dfrac{3}{8}$ Mädchen.

    \begin{enumerate}[label=\alph*)]
        \item Wie viele Mädchen sind in der Klasse?

        \vspace{0.3cm}
        Rechnung: $\dfrac{3}{8}$ von 24 = $24 : 8 \cdot 3$ = \underline{\hspace{5cm}}

        \vspace{0.3cm}
        Antwort: In der Klasse sind \underline{\hspace{3cm}} Mädchen.

        \vspace{0.5cm}
        \item Wie viele Jungen sind in der Klasse?

        \vspace{0.3cm}
        Rechnung: $24 -$ \underline{\hspace{2cm}} = \underline{\hspace{3cm}}

        \vspace{0.3cm}
        Antwort: In der Klasse sind \underline{\hspace{3cm}} Jungen.
    \end{enumerate}
\end{enumerate}