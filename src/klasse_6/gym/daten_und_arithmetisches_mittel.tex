\subsection*{Teil D: Daten und arithmetisches Mittel (15 Minuten)}

\begin{enumerate}[label=\arabic*.]
    \item \textbf{Arithmetisches Mittel berechnen:}

    Das arithmetische Mittel (Durchschnitt) berechnet man so:

    $\text{Durchschnitt} = \dfrac{\text{Summe aller Werte}}{\text{Anzahl der Werte}}$

    \vspace{0.5cm}

    Beispiel: Noten: 2, 3, 1, 4, 2

    Durchschnitt = $\dfrac{2 + 3 + 1 + 4 + 2}{5} = \dfrac{12}{5} = 2,4$

    \vspace{0.5cm}

    Berechne selbst:

    a) Temperaturen: 18°C, 22°C, 20°C, 16°C, 24°C

    Durchschnitt = $\dfrac{\phantom{00} + \phantom{00} + \phantom{00} + \phantom{00} + \phantom{00}}{\phantom{0}} = \dfrac{\phantom{000}}{\phantom{0}} = $ \underline{\hspace{2cm}}°C

    \vspace{0.5cm}

    b) Punkte in Klassenarbeiten: 45, 38, 42, 40, 35

    Durchschnitt = \underline{\hspace{3cm}} Punkte

    \vspace{0.5cm}

    \item \textbf{Tabelle auswerten:}

    \begin{tabular}{|l|c|c|c|c|c|}
    \hline
    Lieblingsfach & Mathe & Deutsch & Englisch & Sport & Kunst \\
    \hline
    Anzahl Schüler & 8 & 5 & 6 & 10 & 3 \\
    \hline
    \end{tabular}

    a) Wie viele Schüler wurden befragt? \underline{\hspace{3cm}}

    b) Welches Fach ist am beliebtesten? \underline{\hspace{3cm}}

    c) Wie viel Prozent mögen Sport am liebsten? \underline{\hspace{3cm}}

\end{enumerate}