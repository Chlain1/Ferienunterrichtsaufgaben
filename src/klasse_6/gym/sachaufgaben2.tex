\subsection*{Teil D: Sachaufgaben (20 Minuten)}

\begin{enumerate}[label=\arabic*.]
    \item \textbf{Rezept-Aufgabe:}

    Für einen Kuchen braucht man $\dfrac{3}{4}$ Liter Milch. Wie viel Milch braucht man für $\dfrac{1}{2}$ Kuchen?

    Rechnung: $\dfrac{3}{4} \cdot \dfrac{1}{2} = $ \underline{\hspace{3cm}} Liter

    \vspace{0.5cm}

    \item \textbf{Schokolade-Aufgabe:}

    Lisa hat $\dfrac{2}{3}$ einer Tafel Schokolade. Sie möchte sie gleichmäßig auf 4 Freunde aufteilen.

    Wie viel bekommt jeder Freund?

    Rechnung: $\dfrac{2}{3} : 4 = $ \underline{\hspace{3cm}} der ganzen Tafel

    \vspace{0.5cm}

    \item \textbf{Stoff-Aufgabe:}

    Für ein Kleid braucht man $1\dfrac{1}{2}$ Meter Stoff. Wie viel Stoff braucht man für $2\dfrac{1}{3}$ Kleider?

    Hinweis: Wandle zuerst in unechte Brüche um.

    $1\dfrac{1}{2} = \dfrac{\phantom{00}}{\phantom{00}}$ \quad $2\dfrac{1}{3} = \dfrac{\phantom{00}}{\phantom{00}}$

    Rechnung: $\dfrac{\phantom{00}}{\phantom{00}} \cdot \dfrac{\phantom{00}}{\phantom{00}} = $ \underline{\hspace{3cm}} Meter

    \vspace{0.5cm}

    \item \textbf{Geld-Aufgabe:}

    Tom hat 24€. Er gibt $\dfrac{1}{3}$ seines Geldes für ein Buch aus und $\dfrac{1}{4}$ für Süßigkeiten.

    a) Wie viel gibt er für das Buch aus? $24 \cdot \dfrac{1}{3} = $ \underline{\hspace{2cm}} €

    b) Wie viel gibt er für Süßigkeiten aus? $24 \cdot \dfrac{1}{4} = $ \underline{\hspace{2cm}} €

    c) Wie viel Geld hat er noch? \underline{\hspace{3cm}} €

\end{enumerate}
