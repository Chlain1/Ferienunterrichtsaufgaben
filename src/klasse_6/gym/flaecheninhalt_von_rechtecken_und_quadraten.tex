\subsection*{Teil A: Flächeninhalt von Rechtecken und Quadraten (25 Minuten)}

\begin{enumerate}[label=\arabic*.]
    \item \textbf{Formeln wiederholen:}
    \vspace{0.5cm}

    Flächeninhalt Rechteck: $A = $ \underline{\hspace{3cm}}

    Flächeninhalt Quadrat: $A = $ \underline{\hspace{3cm}}

    \vspace{1cm}

    \item \textbf{Berechne den Flächeninhalt:}
    \vspace{0.5cm}

    \begin{tabular}{ll}
        a) Rechteck: $a = 6$ cm, $b = 4$ cm & b) Quadrat: $a = 5$ cm \\[2ex]
        $A = $ \underline{\hspace{4cm}} & $A = $ \underline{\hspace{4cm}} \\[4ex]
        c) Rechteck: $a = 8$ m, $b = 3$ m & d) Quadrat: $a = 7$ m \\[2ex]
        $A = $ \underline{\hspace{4cm}} & $A = $ \underline{\hspace{4cm}}
    \end{tabular}

    \vspace{1cm}

    \item \textbf{Finde die fehlende Seite:}
    \vspace{0.5cm}

    \begin{enumerate}[label=\alph*)]
        \item Rechteck mit $A = 20$ cm², $a = 5$ cm. Wie lang ist $b$?

        $20 = 5 \cdot b$ → $b = $ \underline{\hspace{3cm}}

        \vspace{0.5cm}

        \item Quadrat mit $A = 36$ m². Wie lang ist $a$?

        $36 = a \cdot a$ → $a = $ \underline{\hspace{3cm}}
    \end{enumerate}

\end{enumerate}
