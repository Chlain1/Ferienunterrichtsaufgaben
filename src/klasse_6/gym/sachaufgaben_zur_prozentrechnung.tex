\subsection*{Teil C: Sachaufgaben zur Prozentrechnung (20 Minuten)}

\begin{enumerate}[label=\arabic*.]
    \item \textbf{Rabatt-Aufgabe:}

    Ein Fahrrad kostet normalerweise 240€. Es gibt 15\% Rabatt.

    a) Wie viel € Rabatt bekommt man? \underline{\hspace{3cm}}

    b) Wie viel kostet das Fahrrad nach dem Rabatt? \underline{\hspace{3cm}}

    \vspace{0.5cm}

    \item \textbf{Klassenarbeit-Aufgabe:}

    In einer Klassenarbeit haben von 30 Schülern 6 Schüler eine 1 geschrieben.

    a) Wie viel Prozent der Schüler haben eine 1? \underline{\hspace{3cm}}

    b) 40\% der Schüler haben eine 2 geschrieben. Wie viele Schüler sind das? \underline{\hspace{3cm}}

    \vspace{0.5cm}

    \item \textbf{Sportverein-Aufgabe:}

    In einem Sportverein sind 25\% der 160 Mitglieder Kinder.

    a) Wie viele Kinder sind im Verein? \underline{\hspace{3cm}}

    b) 30\% sind Jugendliche. Wie viele Jugendliche sind das? \underline{\hspace{3cm}}

    c) Der Rest sind Erwachsene. Wie viele Erwachsene sind das? \underline{\hspace{3cm}}

\end{enumerate}