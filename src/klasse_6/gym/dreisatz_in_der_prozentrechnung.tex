\subsection*{Teil B: Dreisatz in der Prozentrechnung (25 Minuten)}

\begin{enumerate}[label=\arabic*.]
    \item \textbf{Prozentwert berechnen:}

    \textit{Beispiel:} In einer Klasse mit 24 Schülern sind 25\% Mädchen. Wie viele Mädchen sind das?

    \begin{tabular}{|l|l|}
        \hline
        Schüler & Prozent \\
        \hline
        24 & 100\% \\
        ? & 25\% \\
        \hline
    \end{tabular}

    $\dfrac{24 \cdot 25}{100} = \dfrac{600}{100} = 6$ Mädchen

    \vspace{0.5cm}

    Löse selbst:

    a) Von 200 Äpfeln sind 15\% faul. Wie viele Äpfel sind faul?

    \begin{tabular}{|l|l|}
        \hline
        Äpfel & Prozent \\
        \hline
        \phantom{000} & 100\% \\
        ? & \phantom{00}\% \\
        \hline
    \end{tabular}

    Antwort: \underline{\hspace{3cm}} Äpfel

    \vspace{0.5cm}

    b) Ein Pullover kostet 80€. Im Sale gibt es 20\% Rabatt. Wie viel spart man?

    Antwort: \underline{\hspace{3cm}} €

    \vspace{1cm}

    \item \textbf{Grundwert berechnen:}

    30\% entsprechen 15 Schülern. Wie viele Schüler sind insgesamt in der Klasse?

    \begin{tabular}{|l|l|}
        \hline
        Schüler & Prozent \\
        \hline
        15 & 30\% \\
        ? & 100\% \\
        \hline
    \end{tabular}

    $\dfrac{15 \cdot 100}{30} = \dfrac{1500}{30} = $ \underline{\hspace{2cm}} Schüler

    \vspace{0.5cm}

    Löse selbst:

    20\% entsprechen 8 Büchern. Wie viele Bücher sind es insgesamt?

    Antwort: \underline{\hspace{3cm}} Bücher

\end{enumerate}