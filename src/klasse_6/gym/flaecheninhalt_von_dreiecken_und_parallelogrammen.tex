\subsection*{Teil B: Flächeninhalt von Dreiecken und Parallelogrammen (25 Minuten)}

\begin{enumerate}[label=\arabic*.]
    \item \textbf{Formeln wiederholen:}
    \vspace{0.5cm}

    Flächeninhalt Dreieck: $A = $ \underline{\hspace{4cm}}

    Flächeninhalt Parallelogramm: $A = $ \underline{\hspace{4cm}}

    \vspace{1cm}

    \item \textbf{Berechne den Flächeninhalt:}
    \vspace{0.5cm}

    \begin{tabular}{ll}
        a) Dreieck: $g = 8$ cm, $h = 6$ cm & b) Parallelogramm: $a = 7$ cm, $h_a = 4$ cm \\[2ex]
        $A = \dfrac{1}{2} \cdot 8 \cdot 6 = $ \underline{\hspace{3cm}} & $A = 7 \cdot 4 = $ \underline{\hspace{3cm}} \\[4ex]
        c) Dreieck: $g = 10$ m, $h = 5$ m & d) Parallelogramm: $a = 9$ m, $h_a = 3$ m \\[2ex]
        $A = $ \underline{\hspace{4cm}} & $A = $ \underline{\hspace{4cm}}
    \end{tabular}

    \vspace{1cm}

    \item \textbf{Sachaufgabe:}

    Ein dreieckiges Verkehrsschild hat eine Grundseite von 60 cm und eine Höhe von 50 cm.

    Wie groß ist der Flächeninhalt? \underline{\hspace{4cm}}

\end{enumerate}