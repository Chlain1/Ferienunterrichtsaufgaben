\subsection*{Teil E: Komplexe Sachaufgaben (10 Minuten)}

\begin{enumerate}[label=\arabic*.]
    \item \textbf{Swimming-Pool:}

    Ein rechteckiger Pool ist 8 m lang, 5 m breit und 1,5 m tief. Er ist zu $\dfrac{3}{4}$ mit Wasser gefüllt.

    a) Wie groß ist das Volumen des Pools? \underline{\hspace{3cm}}

    b) Wie viele Liter Wasser sind im Pool? \underline{\hspace{3cm}}

    \vspace{0.5cm}

    \item \textbf{Klassenfahrt:}

    Für eine Klassenfahrt werden 480€ gesammelt. $\dfrac{2}{5}$ des Geldes ist für die Unterkunft, 30\% für das Essen und der Rest für Ausflüge.

    a) Wie viel Geld ist für die Unterkunft? \underline{\hspace{3cm}}

    b) Wie viel Geld ist für das Essen? \underline{\hspace{3cm}}

    c) Wie viel Geld bleibt für Ausflüge? \underline{\hspace{3cm}}

    \vspace{0.5cm}

    \item \textbf{Buchseiten:}

    Lisa liest ein Buch mit 240 Seiten. Am ersten Tag liest sie $\dfrac{1}{6}$ des Buches, am zweiten Tag 25\% und am dritten Tag 45 Seiten.

    a) Wie viele Seiten liest sie am ersten Tag? \underline{\hspace{3cm}}

    b) Wie viele Seiten liest sie am zweiten Tag? \underline{\hspace{3cm}}

    c) Wie viele Seiten hat sie insgesamt gelesen? \underline{\hspace{3cm}}

    d) Wie viele Seiten muss sie noch lesen? \underline{\hspace{3cm}}

\end{enumerate}