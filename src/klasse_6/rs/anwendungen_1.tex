\subsection*{Teil D: Anwendungsaufgaben (15 Minuten)}

\begin{enumerate}[label=\arabic*.]
    \item \textbf{Spielfeld:}
    Ein rechteckiges Fußballfeld ist 100 m lang und 64 m breit.
    \vspace{0.5cm}

        \begin{enumerate}[label=\alph*)]
        \item Berechne die Fläche des Spielfeldes:

        $A = \underline{\hspace{2cm}} \cdot \underline{\hspace{2cm}} = \underline{\hspace{4cm}}$

        \vspace{0.5cm}

        \item Wie lang ist der Umfang des Feldes?

        $U = 2 \cdot a + 2 \cdot b = 2 \cdot \underline{\hspace{2cm}} + 2 \cdot \underline{\hspace{2cm}} = \underline{\hspace{4cm}}$
    \end{enumerate}

    \vspace{1cm}

    \item \textbf{Verkehrsschild:}
    Ein dreieckiges Verkehrsschild hat eine Grundseite von 60 cm und eine Höhe von 52 cm.
    \vspace{0.5cm}

    $A = \dfrac{1}{2} \cdot g \cdot h = \dfrac{1}{2} \cdot \underline{\hspace{2cm}} \cdot \underline{\hspace{2cm}} = \dfrac{\phantom{0000}}{2} = \underline{\hspace{3cm}}$

    \vspace{1cm}

    \item \textbf{Zimmer streichen:}
    Ein quadratisches Zimmer hat eine Seitenlänge von 4,5 m. Der Boden soll gestrichen werden.
    \vspace{0.5cm}

    \begin{enumerate}[label=\alph*)]
        \item Wie groß ist die zu streichende Fläche?

        $A = a^2 = \underline{\hspace{2cm}} \cdot \underline{\hspace{2cm}} = \underline{\hspace{4cm}}$

        \vspace{0.5cm}

        \item 1 Liter Farbe reicht für 8 m². Wie viel Farbe wird benötigt?

        Rechnung: $\underline{\hspace{3cm}} : 8 = \underline{\hspace{3cm}}$ Liter
    \end{enumerate}
\end{enumerate}