\subsection*{Teil D: Gleichungen aus Texten (15 Minuten)}

\begin{enumerate}[label=\arabic*.]
    \item \textbf{Stelle Gleichungen auf und löse sie:}
    \vspace{0.5cm}

    \begin{enumerate}[label=\alph*)]
        \item Ich denke mir eine Zahl, addiere 9 dazu und erhalte 17. Wie heißt meine Zahl?

        Gleichung: $x + 9 = 17$

        Lösung: $x + 9 - 9 = 17 - 9$

        $x = \underline{\hspace{3cm}}$

        \vspace{0.5cm}

        \item Das Vierfache einer Zahl ist 24. Wie heißt die Zahl?

        Gleichung: $\underline{\hspace{3cm}} \cdot x = \underline{\hspace{3cm}}$

        Lösung: $\underline{\hspace{3cm}} \cdot x : \underline{\hspace{2cm}} = \underline{\hspace{3cm}} : \underline{\hspace{2cm}}$

        $x = \underline{\hspace{3cm}}$

        \vspace{0.5cm}

        \item Wenn ich von einer Zahl 11 subtrahiere, erhalte ich 8. Wie heißt die Zahl?

        Gleichung: $x - \underline{\hspace{2cm}} = \underline{\hspace{2cm}}$

        Lösung: $x - \underline{\hspace{2cm}} + \underline{\hspace{2cm}} = \underline{\hspace{2cm}} + \underline{\hspace{2cm}}$

        $x = \underline{\hspace{3cm}}$

        \vspace{0.5cm}

        \item Ein Rechteck hat eine Länge von 8 cm und einen Flächeninhalt von 48 cm². Wie breit ist das Rechteck?

        Gleichung: $8 \cdot x = 48$

        Lösung: $8 \cdot x : \underline{\hspace{2cm}} = 48 : \underline{\hspace{2cm}}$

        $x = \underline{\hspace{3cm}}$ cm

        \vspace{0.5cm}

        \item Peter hat einige Murmeln. Wenn er 7 Murmeln dazubekommt, hat er 23 Murmeln. Wie viele Murmeln hatte er anfangs?

        Gleichung: $x + \underline{\hspace{2cm}} = \underline{\hspace{2cm}}$

        Lösung: $x = \underline{\hspace{3cm}}$ Murmeln
    \end{enumerate}
\end{enumerate}