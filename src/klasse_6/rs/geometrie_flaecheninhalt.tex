\subsection*{Teil B: Geometrie - Flächeninhalt (25 Minuten)}

\begin{enumerate}[label=\arabic*.]
    \item \textbf{Flächeninhalt berechnen:}
    \vspace{0.5cm}

    \begin{enumerate}[label=\alph*)]
        \item Rechteck mit $a = 7{,}5\,\text{cm}$ und $b = 4\,\text{cm}$:

        $A = a \cdot b = 7{,}5\,\text{cm} \cdot 4\,\text{cm} = \underline{\hspace{4cm}}$

        \vspace{0.5cm}

        \item Dreieck mit $g = 8\,\text{cm}$ und $h = 6\,\text{cm}$:

        $A = \dfrac{1}{2} \cdot g \cdot h = \dfrac{1}{2} \cdot 8\,\text{cm} \cdot 6\,\text{cm} = \dfrac{\phantom{00}}{2} = \underline{\hspace{3cm}}$

        \vspace{0.5cm}

        \item Parallelogramm mit $a = 9\,\text{cm}$ und $h_a = 5\,\text{cm}$:

        $A = a \cdot h_a = \underline{\hspace{2cm}} \cdot \underline{\hspace{2cm}} = \underline{\hspace{4cm}}$

        \vspace{0.5cm}

        \item Ein Quadrat hat den Flächeninhalt $36\,\text{cm}^2$. Wie lang ist eine Seite?

        $A = a^2 = 36\,\text{cm}^2$

        $a = \sqrt{36\,\text{cm}^2} = \underline{\hspace{3cm}}$, denn $6^2 = 6 \cdot 6 = 36$
    \end{enumerate}

    \vspace{1cm}

    \item \textbf{Symmetrie:}
    \vspace{0.5cm}

    \begin{enumerate}[label=\alph*)]
        \item Nenne zwei Symmetrieachsen eines Rechtecks:

        1. \underline{\hspace{4cm}} \hspace{1cm} 2. \underline{\hspace{4cm}}

        \vspace{0.5cm}

        \item Wie viele Symmetrieachsen hat ein Quadrat? \underline{\hspace{3cm}}

        \vspace{0.5cm}

        \item Ist ein Parallelogramm achsensymmetrisch? \underline{\hspace{3cm}}
    \end{enumerate}
\end{enumerate}
