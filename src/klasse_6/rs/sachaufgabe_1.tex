\subsection*{Teil D: Sachaufgaben (15 Minuten)}

\begin{enumerate}[label=\arabic*.]
    \item \textbf{Lisa und Tom teilen sich einen Kuchen:}
    \vspace{0.5cm}

    Lisa isst $\dfrac{1}{4}$ des Kuchens, Tom isst $\dfrac{2}{8}$ des Kuchens.

    \vspace{0.5cm}
    \begin{enumerate}[label=\alph*)]
        \item Wandle Toms Anteil in Vierteln um: $\dfrac{2}{8} = \dfrac{\phantom{00}}{\phantom{00}}$
        \vspace{0.5cm}
        \item Wer hat mehr gegessen? \underline{\hspace{6cm}}
        \vspace{0.5cm}
        \item Wie viel haben beide zusammen gegessen? 

        Rechnung: $\dfrac{1}{4} + \dfrac{2}{8} = \dfrac{1}{4} + \dfrac{\phantom{00}}{4} = \dfrac{\phantom{00}}{4}$ = \underline{\hspace{3cm}}
    \end{enumerate}

    \vspace{1cm}

    \item \textbf{Wasserglas-Aufgabe:}
    \vspace{0.5cm}

    Ein Wasserglas ist zu $\dfrac{3}{5}$ gefüllt.

    \vspace{0.5cm}
    \begin{enumerate}[label=\alph*)]
        \item Wandle in Dezimalzahl um: $\dfrac{3}{5} = 3 : 5 =$ \underline{\hspace{3cm}}
        \vspace{0.5cm}
        \item Wandle in Prozent um: \underline{\hspace{3cm}} $\%$
        \vspace{0.5cm}
        \item Wie viel fehlt noch bis das Glas voll ist?

        Rechnung: $1 - \dfrac{3}{5} = \dfrac{5}{5} - \dfrac{3}{5} = \dfrac{\phantom{00}}{5}$ = \underline{\hspace{3cm}}
    \end{enumerate}
\end{enumerate}