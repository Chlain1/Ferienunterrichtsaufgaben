\subsection*{Teil D: Direkte Proportionalität und Dreisatz (15 Minuten)}

\begin{enumerate}[label=\arabic*.]
    \item \textbf{Erkenne die Proportionalität:}
    Kreuze an, welche Zuordnungen direkt proportional sind:
    \vspace{0.5cm}

    \begin{tabular}{|l|c|c|}
        \hline
        \textbf{Zuordnung} & \textbf{direkt proportional} & \textbf{nicht proportional} \\
        \hline
        Anzahl der Äpfel $\rightarrow$ Preis & $\square$ & $\square$ \\
        \hline
        Alter einer Person $\rightarrow$ Körpergröße & $\square$ & $\square$ \\
        \hline
        Geschwindigkeit $\rightarrow$ Zeit für gleiche Strecke & $\square$ & $\square$ \\
        \hline
        Benzinverbrauch $\rightarrow$ gefahrene Kilometer & $\square$ & $\square$ \\
        \hline
        Anzahl der Hefte $\rightarrow$ Gesamtpreis & $\square$ & $\square$ \\
        \hline
    \end{tabular}

    \vspace{1cm}

    \item \textbf{Löse mit dem Dreisatz:}
    \vspace{0.5cm}

    \begin{enumerate}[label=\alph*)]
        \item 4 kg Äpfel kosten 8 Euro. Was kosten 7 kg Äpfel?

        \begin{tabular}{l|l}
            4 kg & 8 Euro \\
            1 kg & $8 : 4 = \underline{\hspace{2cm}}$ Euro \\
            7 kg & $\underline{\hspace{2cm}} \cdot 7 = \underline{\hspace{2cm}}$ Euro \\
        \end{tabular}

        Antwort: \underline{\hspace{4cm}}

        \vspace{1cm}

        \item Ein Auto verbraucht auf 100 km 7 Liter Benzin. Wie viel verbraucht es auf 350 km?

        \begin{tabular}{l|l}
            100 km & 7 l \\
            1 km & $7 : 100 = \underline{\hspace{2cm}}$ l \\
            350 km & $\underline{\hspace{2cm}} \cdot 350 = \underline{\hspace{2cm}}$ l \\
        \end{tabular}

        Antwort: \underline{\hspace{4cm}}

        \vspace{1cm}

        \item 5 Hefte kosten 12{,}50 Euro. Was kosten 8 Hefte?

        \begin{tabular}{l|l}
            5 Hefte & 12{,}50 Euro \\
            1 Heft & $12{,}50 : 5 = \underline{\hspace{2cm}}$ Euro \\
            8 Hefte & $\underline{\hspace{2cm}} \cdot 8 = \underline{\hspace{2cm}}$ Euro \\
        \end{tabular}

        Antwort: \underline{\hspace{4cm}}
    \end{enumerate}
\end{enumerate}