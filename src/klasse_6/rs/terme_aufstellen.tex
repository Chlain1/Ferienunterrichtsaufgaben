\subsection*{Teil B: Terme aufstellen (20 Minuten)}

\begin{enumerate}[label=\arabic*.]
    \item \textbf{Stelle zu den folgenden Situationen Terme auf:}
    \vspace{0.5cm}

    \begin{enumerate}[label=\alph*)]
        \item Lisa hat $x$ Euro. Sie bekommt 8 Euro dazu. Wie viel hat sie dann?

        Term: \underline{\hspace{4cm}}

        \vspace{0.5cm}

        \item Tom hat 20 Äpfel und gibt $x$ Äpfel weg. Wie viele bleiben übrig?

        Term: \underline{\hspace{4cm}}

        \vspace{0.5cm}

        \item Ein Rechteck hat die Länge $x$ cm und die Breite 6 cm. Wie groß ist der Flächeninhalt?

        Term: \underline{\hspace{4cm}}

        \vspace{0.5cm}

        \item Anna kauft 5 Hefte. Jedes Heft kostet $x$ Euro. Was zahlt sie insgesamt?

        Term: \underline{\hspace{4cm}}

        \vspace{0.5cm}

        \item Ein Quadrat hat die Seitenlänge $x$ cm. Wie groß ist der Umfang?

        Term: \underline{\hspace{4cm}}
    \end{enumerate}

    \vspace{1cm}

    \item \textbf{Setze die Werte ein und berechne:}
    \textit{Aus Aufgabe 1a: Lisa hat $x$ Euro und bekommt 8 Euro dazu. Term: $x + 8$}
    \vspace{0.5cm}

    \begin{enumerate}[label=\alph*)]
        \item Wenn $x = 5$: Lisa hat dann $5 + 8 = \underline{\hspace{3cm}}$ Euro

        \vspace{0.5cm}

        \item Wenn $x = 12$: Lisa hat dann $12 + 8 = \underline{\hspace{3cm}}$ Euro

        \vspace{0.5cm}

        \item Wenn $x = 7{,}50$: Lisa hat dann $7{,}50 + 8 = \underline{\hspace{3cm}}$ Euro
    \end{enumerate}
\end{enumerate}