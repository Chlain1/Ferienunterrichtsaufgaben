\subsection*{Teil E: Proportionalität (10 Minuten)}

\begin{enumerate}[label=\arabic*.]
    \item \textbf{Dreisatz anwenden:}
    \vspace{0.5cm}

    6 Brötchen kosten 3{,}60 Euro. Was kosten 10 Brötchen?

    \begin{tabular}{l|l}
        6 Brötchen & 3{,}60 Euro \\
        1 Brötchen & $3{,}60 : 6 = \underline{\hspace{2cm}}$ Euro \\
        10 Brötchen & $\underline{\hspace{2cm}} \cdot 10 = \underline{\hspace{2cm}}$ Euro \\
    \end{tabular}

    Antwort: \underline{\hspace{4cm}}

    \vspace{1cm}

    \item \textbf{Geschwindigkeit:}
    Ein Zug fährt in 2 Stunden 180 km. Wie weit fährt er in 3 Stunden bei gleicher Geschwindigkeit?

    Geschwindigkeit: $180\,\text{km} : 2\,\text{h} = \underline{\hspace{3cm}}$ km/h

    In 3 Stunden: $\underline{\hspace{3cm}} \cdot 3 = \underline{\hspace{4cm}}$
\end{enumerate}