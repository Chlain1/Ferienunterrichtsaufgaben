\subsection*{Teil A: Brüche verstehen und darstellen (20 Minuten)}

\begin{enumerate}[label=\arabic*.]
    \item \textbf{Welcher Bruch ist dargestellt? Schreibe auf:}
    \vspace{0.5cm}

    \textit{Stelle dir vor: Ein Kreis ist in 4 gleiche Teile geteilt, 3 sind gefärbt}

    \begin{tabular}{ll}
        a) 3 von 4 Teilen = $\dfrac{\phantom{00}}{\phantom{00}}$ & b) 1 von 2 Teilen = $\dfrac{\phantom{00}}{\phantom{00}}$ \\[2ex]
        c) 5 von 8 Teilen = $\dfrac{\phantom{00}}{\phantom{00}}$ & d) 2 von 3 Teilen = $\dfrac{\phantom{00}}{\phantom{00}}$
    \end{tabular}

    \vspace{1cm}

    \item \textbf{Erweitere die Brüche:} 
    \vspace{0.5cm}
    
    \begin{tabular}{ll}
        a) $\dfrac{1}{2} = \dfrac{1 \cdot 2}{2 \cdot 2} = \dfrac{\phantom{00}}{4}$ & b) $\dfrac{1}{3} = \dfrac{1 \cdot 2}{3 \cdot 2} = \dfrac{\phantom{00}}{6}$ \\[4ex]
        c) $\dfrac{2}{5} = \dfrac{2 \cdot 2}{5 \cdot 2} = \dfrac{\phantom{00}}{10}$ & d) $\dfrac{3}{4} = \dfrac{3 \cdot 3}{4 \cdot 3} = \dfrac{\phantom{00}}{12}$
    \end{tabular}

    \vspace{1cm}

    \item \textbf{Kürze die Brüche so weit wie möglich:} 
    \vspace{0.5cm}

    \begin{tabular}{ll}
        a) $\dfrac{6}{8} = \dfrac{6:2}{8:2} = \dfrac{\phantom{00}}{\phantom{00}}$ & b) $\dfrac{9}{12} = \dfrac{9:3}{12:3} = \dfrac{\phantom{00}}{\phantom{00}}$ \\[4ex]
        c) $\dfrac{10}{15} = \dfrac{10:5}{15:5} = \dfrac{\phantom{00}}{\phantom{00}}$ & d) $\dfrac{14}{21} = \dfrac{14:7}{21:7} = \dfrac{\phantom{00}}{\phantom{00}}$
    \end{tabular}
\end{enumerate}