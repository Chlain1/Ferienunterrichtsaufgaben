\subsection*{Teil A: Bruchrechnung und Dezimalzahlen (20 Minuten)}

\begin{enumerate}[label=\arabic*.]
    \item \textbf{Grundrechenarten mit Brüchen:}
    \vspace{0.5cm}

    \begin{tabular}{ll}
        a) $\dfrac{2}{5} + \dfrac{1}{5} = \dfrac{2+1}{5} = \dfrac{\phantom{00}}{5}$ = \underline{\hspace{3cm}} &
        b) $\dfrac{5}{6} - \dfrac{2}{6} = \dfrac{5-2}{6} = \dfrac{\phantom{00}}{6}$ = \underline{\hspace{3cm}} \\[4ex]
        c) $\dfrac{1}{2} + \dfrac{1}{4} = \dfrac{\phantom{00}}{4} + \dfrac{1}{4} = \dfrac{\phantom{00}}{4}$ = \underline{\hspace{3cm}} &
        d) $\dfrac{3}{4} - \dfrac{1}{2} = \dfrac{3}{4} - \dfrac{\phantom{00}}{4} = \dfrac{\phantom{00}}{4}$ = \underline{\hspace{3cm}}
    \end{tabular}

    \vspace{1cm}

    \item \textbf{Umwandlungen:}
    \vspace{0.5cm}

    \begin{enumerate}[label=\alph*)]
        \item Wandle um: $0{,}75 = \dfrac{75}{100} = \dfrac{\phantom{00}}{\phantom{00}}$ (als gekürzten Bruch)

        \vspace{0.5cm}

        \item Wandle um: $\dfrac{3}{5} = 3 : 5 = \underline{\hspace{3cm}}$ (als Dezimalzahl)

        \vspace{0.5cm}

        \item Vergleiche: $\dfrac{3}{4}$ \underline{\hspace{1cm}} $\dfrac{2}{3}$
        \textit{Tipp: $\dfrac{3}{4} = 0{,}75$ und $\dfrac{2}{3} = 0{,}66...$}

        \vspace{0.5cm}

        \item Kürze: $\dfrac{12}{18} = \dfrac{12:6}{18:6} = \dfrac{\phantom{00}}{\phantom{00}}$
    \end{enumerate}
\end{enumerate}