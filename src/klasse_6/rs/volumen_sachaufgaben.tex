\subsection*{Teil C: Volumen-Sachaufgaben (20 Minuten)}

\begin{enumerate}[label=\arabic*.]
    \item \textbf{Schwimmbecken:}
    Ein Schwimmbecken ist 12 m lang, 6 m breit und 1{,}5 m tief.
    \vspace{0.5cm}

    \begin{enumerate}[label=\alph*)]
        \item Wie viel Wasser passt hinein?

        $V = a \cdot b \cdot c = \underline{\hspace{2cm}} \cdot \underline{\hspace{2cm}} \cdot \underline{\hspace{2cm}} = \underline{\hspace{4cm}}$

        \vspace{0.5cm}

        \item Wie viele Liter sind das?

        $\underline{\hspace{3cm}}\,\text{m}^3 = \underline{\hspace{3cm}} \cdot 1000\,\text{l} = \underline{\hspace{4cm}}$ Liter
    \end{enumerate}

    \vspace{1cm}

    \item \textbf{Geschenkkarton:}
    Ein Karton ist 30 cm lang, 20 cm breit und 15 cm hoch.
    \vspace{0.5cm}

    $V = \underline{\hspace{2cm}} \cdot \underline{\hspace{2cm}} \cdot \underline{\hspace{2cm}} = \underline{\hspace{4cm}}$

    \vspace{1cm}

    \item \textbf{Aquarium:}
    Ein würfelförmiges Aquarium hat eine Kantenlänge von 40 cm.
    \vspace{0.5cm}

    \begin{enumerate}[label=\alph*)]
        \item Wie groß ist das Volumen?

        $V = a^3 = (\underline{\hspace{2cm}})^3 = \underline{\hspace{4cm}}$

        \vspace{0.5cm}

        \item Wie viele Liter Wasser passen hinein?

        $\underline{\hspace{3cm}}\,\text{cm}^3 = \underline{\hspace{3cm}} : 1000\,\text{l} = \underline{\hspace{3cm}}$ Liter
    \end{enumerate}
\end{enumerate}